\documentclass{article}
\usepackage{amsmath}
\usepackage{amsfonts}
\usepackage{amssymb,amsthm}
\usepackage{color}
\usepackage[margin=1in]{geometry}
\usepackage{hyperref}
\usepackage{graphicx}
\usepackage{verbatim}

\numberwithin{equation}{section}
\newtheorem{theorem}[equation]{Theorem}
\newtheorem{proposition}[equation]{Proposition}
\newtheorem{lemma}[equation]{Lemma}
\newtheorem{corollary}[equation]{Corollary}
\newtheorem{conjecture}[equation]{Conjecture}
\newtheorem{problem}[equation]{Problem}
\newtheorem{remark}[equation]{Remark}
\newtheorem{definition}[equation]{Definition}
\newenvironment{defn}[1][]{%
    \begin{definition}[#1]\pushQED{\qed}}{\popQED \end{definition}}
\parindent=0in
\parskip=0.15in

\newcommand{\xx}[1]{\textcolor{blue}{#1}}
\newcommand{\zb}{\overline{z}}
\newcommand{\cc}[1]{\overline{#1}}
\newcommand{\m}[1]{\left|#1\right|}
\newcommand{\Res}{\operatorname{Res}}
\newcommand{\p}[1]{\left(#1\right)}
\newcommand{\rank}{\operatorname{rank}}
\newcommand{\sgn}{\operatorname{sgn}}
\newcommand{\D}{\triangle}
\newcommand{\ZZ}{\mathbb{Z}}
\newcommand{\f}[1]{\sum_{n=-\infty}^\infty #1_ne^{inx}}
\newcommand{\fn}[1]{\sum_{n=-\infty}^\infty #1e^{inx}}
\newcommand{\BB}{\mathcal{B}}
\newcommand{\M}{\mathcal{M}}
\newcommand{\mm}[1]{\m{\m{#1}}}
\newcommand{\ang}[1]{\left\langle #1 \right\rangle}
\newcommand{\h}{\mathfrak{h}}
\newcommand{\GL}{{\rm GL}}
\newcommand{\HH}{{\rm H}}
\newcommand{\arxiv}[1]{\href{http://arxiv.org/abs/#1}{{\tt arXiv:#1}}}
\newcommand{\Sym}{\operatorname{Sym}}

\title{Representations of rational Cherednik algebras of $\Sigma_n$ in positive characteristic}

\author{Sheela Devadas}

\date{\today}

\begin{document}


\maketitle

\begin{abstract}
We study lowest-weight irreducible representations of rational Cherednik algebras associated to the symmetric group $\Sigma_n$ in characteristic $p$ when $p$ divides $n$, $\tau$ is trivial, $\hbar=1$, and $c$ is generic. We describe the generators of kernel of the contravariant bilinear form on the Verma module in terms of a formal power series that is related to the residues that generate the kernel in the characteristic $0$ case; we give a formula for the Hilbert series of the irreducible representation.
\end{abstract}

\section{Introduction and Definitions}

In this paper we study lowest-weight representations of rational Cherednik algebras associated to the symmetric group $\Sigma_n$ in characteristic $p$ dividing $n$.

%motivation, history, etc.

Given a vector space $\h$, an element $s \in \GL(\h)$ is an {\it reflection} if it has finite order and $\rank(1-s)=1$. A finite subgroup of $\GL(\h)$ that is generated by reflections is a {\it reflection group}. In particular the symmetric group $\Sigma_n$ in $n$ variables is a reflection group (the reflections are the transpositions). 

Given a reflection group $G \subset \GL(\h)$ and a vector space $\h$ over a field $k$, we let $\mathcal{S}$ be the set of reflections in $G$. For each $s \in \mathcal{S}$ we assign a vector $\alpha_s \in \h^*$ spanning the image of $1-s$, and choose $\alpha_s^\vee \in \h$ so that $(1-s)x=\langle \alpha_s^\vee,x\rangle \alpha_s$ for all $x \in \h^*$, where $\langle \cdot,\cdot\rangle$ indicates the pairing between $\h$ and $\h^*$. We choose $\hbar \in k$ a number and $c: \mathcal{S} \to k$ a function so that $c(s)=c(s')$ whenever $s,s'$ are conjugate. Let $\overline{c}$ be the function defined by $\overline{c}(s)=c(s^{-1})$. 

Let $T(\h \oplus \h^*)$ be the tensor algebra of $\h \oplus \h^*$. Then we define the {\it rational Cherednik algebra $\HH_{\hbar,c}(G,\h)$} as the quotient of $k[G] \ltimes T(\h \oplus \h^*)$ by the following relations for all $x,x' \in \h^*, y,y' \in \h$:
\begin{align*}
[x,x']=0, \quad [y,y' = 0], \quad [y,x] = \hbar\langle y,x\rangle - \sum_{s \in \mathcal{S}} c(s)\langle y,\alpha_s\rangle\langle \alpha_s^\vee,x\rangle s.
\end{align*}
We can give $\HH_{\hbar,c}(G,\h)$ a $\mathbb{Z}$-grading by setting $\deg x=1$ for $x \in \h^*$, $\deg y = -1$ for $y \in \h$, and $\deg g=0$ for $g \in k[G]$. We get the PBW-type decomposition $\HH_{\hbar,c}(G,\h)=\Sym(\h) \otimes_k k[G] \otimes_k \Sym(\h^*)$ (\cite{EM}, section 3.2). 

In general, for any $\alpha \ne 0$, $\HH_{\hbar,c}(G,\h)\simeq \HH_{\alpha\hbar,\alpha c}(G,\h)$. Then we can assume $\hbar=0$ or $\hbar = 1$. 

Let $\tau$ be a representation of $G$. The {\it Verma module} $M_{\hbar,c}(G,\h,\tau)$ is defined as $\HH_{\hbar,c}(G,\h) \otimes_{k[G] \ltimes \Sym(\h)} \tau$. Using the PBW decomposition of the Cherednik algebra, we see that $M_{\hbar,c}(G,\h,\tau)=\Sym(\h^*) \otimes_k \tau$ as a $k$-vector space; we can give this a $\mathbb{Z}$-grading in an obvious way. 

As described in section 2.5 of \cite{BC1}, $M_{\hbar,c}(G,\h,\tau)$ has a unique maximal graded proper submodule $J_{\hbar,c}(G,\h,\tau)$ which can be realized as the kernel of the contravariant form $\beta_c: M_{\hbar,c}(G,\h,\tau) \times M_{\hbar,\overline{c}}(G,\h^*,\tau^*) \to k$; $\beta_c$ can be characterized by the property that for all $x \in \h^*, y \in \h, f \in M_{\hbar,c}(G,\h,\tau), g \in  M_{\hbar,\overline{c}}(G,\h^*,\tau^*), v \in \tau, w \in \tau^*$:
\begin{align*}
\beta_c(xf,g)=\beta_c(f,xg), \quad \beta_c(f,yg) = \beta_c(yf,g), \quad \beta_c(v,w) = \langle v,w\rangle.
\end{align*}
The quotient $L_{\hbar,c}(G,\h,\tau) = M_{\hbar,c}(G,\h,\tau)/J_{\hbar,c}(G,\h,\tau)$ is a finite-dimensional irreducible $\mathbb{Z}$-graded representation of $\HH_{\hbar,c}(G,\h)$.

To understand the action of $\HH_{\hbar,c}(G,\h)$ on $M_{\hbar,c}(G,\h,\tau)$, we can use the PBW decompositions. The action of $\Sym(\h^*)$ on $M_{\hbar,c}(G,\h,\tau)=\Sym(h^*)\otimes_k \tau$ is by left multiplication; $k[G]$ acts by the diagonal action, and $\Sym(\h)$ acts via {\it Dunkl operators}. For $y \in \h$, the Dunkl operator $D_y$ acts on $M_{\hbar,c}(G,\h,\tau)$ by:
\begin{align*}
D_y(f \otimes v) = \hbar \partial_y f \otimes v - \sum_{s \in \mathcal{S}} c(s) \frac{ ( y, \alpha_s )}{\alpha_s} (1-s). f \otimes s.v.
\end{align*}
Throughout the paper we let $G=\Sigma_n$ and $\tau$ the trivial representation. There is only one conjugacy class of reflections in $\Sigma_n$, so $c$ is an element of $k$ for our purposes. We call $c$ {\it generic} if we do not specify its value. We will be concerned with the case $\hbar=1$ and $c$ generic in this paper. (Note that in particular this case means that $\overline{c}=c$, since there is only one conjugacy class of reflections.) The characteristic of the field $k$ is $p>0$. We let $V$ be the vector space spanned by $y_0,\dots,y_{n-1}$ and let $\h$ be the subspace spanned by $y_i-y_j$ for $i \ne j$; $\Sigma_n$ acts by permuting indices. Then if $x_0,\dots,x_{n-1}$ is the dual basis for $V^*$, we see that $\h^*$ is the span of $x_0,\dots,x_{n-1}$ under the relation $x_0+\dots+x_{n-1}=0$; alternatively we can consider $\h^*$ as the span of $x_0,\dots,x_{n-2}$ with $x_{n-1}$ defined as $-x_0-\dots-x_{n-2}$. For a transposition $s_{ij} \in \Sigma_n$ with $i<j$, we let the corresponding vector $\alpha_{s_{ij}} \in \h^*$ be $x_i-x_j$.

In this case, since $\tau$ is trivial, $M_{\hbar,c}(G,\h,\tau)$ is a polynomial ring $k[x_0,\dots,x_{n-2}]$; we call this polynomial ring $A$. Similarly we refer to $J_{\hbar,c}(G,\h,\tau)$ as $J$, which is an ideal in $A$, and $A/J$ is the irreducible representation of the Cherednik algebra we desire to find. Because of the definition of the contravariant form $\beta_c$, showing that an element $f$ of $A$ is in the kernel of $\beta_c$ is equivalent to showing that the Dunkl operators corresponding to the basis elements of $\h$ annihilate $f$. If $D_i$ is the Dunkl operator corresponding to $y_i$, the Dunkl operators for the elements of $\h$ are spanned by $D_i-D_j$ with $i \ne j$. 

We assume $p>2$ since the case $p=2$ is fully characterized in \cite{L}.


\section{Representations of rational Cherednik algebras of $\Sigma_n$ in positive characteristic $p \mid n$}

In this paper we consider the case where the characteristic $p$ of $k$ divides $n$. This is related to the case in characteristic $0$ where $c$ takes the specific value $p/n$, as described in \cite{CE}. In that case the generators of the ideal $J$ were the residues at infinity of $\frac{1}{z-x_a} \prod_{i=0}^{n-1} (z-x_i)^c$ for each $a$. To get a similar result in positive characteristic, we must consider formal power series in $z$ with coefficients from $A$. The formal power series would be $\frac{1}{z^{p+1}}\frac{1}{1-x_az}\prod_{j=0}^{n-1} (1-x_jz)^c$, with the corresponding generator as the coefficient of $1/z$. We simplify and truncate the formal power series so we can define it in positive characteristic.

We let $g=\prod_{j=0}^{n-1} (1-x_jz)$. Let $F_a$ for $a=0,\dots,n-1$ be the formal power series in $z$ defined by $F_a=\frac{1}{1-x_az} \left( \sum_{k=0}^{p-1} \binom{c}{k}(g-1)^k\right)$ where $\binom{c}{k}=\frac{c(c-1)\dots(c-k+1)}{k!}$. 

\begin{proposition}\label{prop:ann} Let $f_a$ be the coefficient of $z^p$ in the power series $F_a$. Then $f_a$ for $a=0,\dots,n-1$ are annihilated by the Dunkl operators. 
\end{proposition}

\begin{proof}
Taking the Dunkl operator of an element of $A$ consists of taking derivatives in the $x_i$, dividing by polynomials in the $x_i$, and letting the symmetric group act on the $x_i$, in addition to linear operations. We see that this means we can apply the Dunkl operators to $F_a$ and check that the coefficient of $z^p$ in the result is $0$ to show that the Dunkl operators annihilate the $f_a$.

We note that each $F_a$ is symmetric in the $x_i$ not including $x_a$, and that for any transposition $s_{ab} \in \Sigma_n$, $s_{ab}F_a=F_b$. Therefore we need only consider the action of the Dunkl operators on $F_0$. We also note that $\mathfrak{h}$ is spanned by $y_i-y_0$ for $0 < i \le n-1$; then using the fact that $F_0$ is symmetric in the $x_i$ with $i \ne 0$, we need only show that $(D_1-D_0)(F_0)$ has $z^p$ coefficient $0$ to show that all of the $f_a$ are annihilated by the Dunkl operators. 

We also note that adding powers $z^k$ with $k > p$ will not change the value of the $z^p$ coefficient in $(D_1-D_0)(F_0)$. In particular, we note that since $x_0+\dots+x_{n-1}=0$ divides the coefficient of $z$ in $g$, we have $z^2 \mid g-1$. Then since $p>2$, we note that $z^{p+1} \mid z^{2p-2} \mid (g-1)^{p-1}$. Therefore we can add multiples of $(g-1)^{p-1}$ when taking the Dunkl operator's action on $F_0$, since even when multipled by another power series it cannot contribute anything to the coefficient of $z^p$. We also note that we can add $n$ times any multiple of $F_0$ since $n \equiv 0 \bmod p$. 

Using the allowed manipulations and the fact that $x_{n-1}=-x_0+\dots-x_{n-2}$, we see that up to the $z^p$ coefficient,

\begin{align*}
\frac{\partial F_0}{\partial x_1}=\left(\frac{zc}{1-x_{n-1}z}-\frac{zc}{1-x_1z}\right)F_0, \quad \frac{\partial F_0}{\partial x_0}=\left(\frac{zc}{1-x_{n-1}z}+\frac{z(1-c)}{1-x_1z}\right)F_0.
\end{align*}

We note that when $0 < i,j$ we have $\frac{1-s_{ij}}{x_i-x_j}\left(F_0\right)=0$. We also see that for $0 < i \le n-1$ we have $\frac{1-s_{ij}}{x_i-x_j}\left(F_0\right)=\frac{z}{1-x_iz}F_0$. 

We also consider $\frac{\partial F_0}{\partial z}$; up to the addition of some multiple of $(g-1)^{p-1}$, this is equal to $\frac{x_0}{1-x_0z}F_0-\sum_{j \ge 0} \frac{-x_jc}{1-x_jz}F_0$. 

Then we see that:

\begin{align*}
(D_1-D_0)(F_0)&=\frac{\partial F_0}{\partial x_1}-\frac{\partial F_0}{\partial x_0}-c\frac{1-s_{01}}{x_1-x_0}(F_0)+c\sum_{j>0}\frac{1-s_{0j}}{x_0-x_j}(F_0)\\
&=\left(\frac{zc}{1-x_{n-1}z}-\frac{zc}{1-x_1z}\right)F_0-\left(\frac{zc}{1-x_{n-1}z}+\frac{z(1-c)}{1-x_1z}\right)F_0+\frac{zc}{1-x_1z}F_0+\sum_{j>0}\frac{zc}{1-x_jz}F_0\\
&=\frac{z(c-1)}{1-x_0z}F_0+\sum_{j>0}\frac{zc}{1-x_jz}F_0\\
&=-\frac{z}{1-x_0z}f_0+\sum_{j} \frac{zc}{1-x_jz}f_0\\
&=-\frac{z}{1-x_0z}f_0+\left(\sum_{j} \frac{zc}{1-x_jz}f_0\right)-nzcf_0\\
&=-zf_0+\frac{z-x_0z^2}{1-x_0z}f_0-\frac{z}{1-x_0z}f_0+\left(\sum_{j} \frac{-zc+x_jcz^2}{1-x_jz}f_0+\frac{zc}{1-x_jz}f_0\right)\\
&=-zf_0+\frac{-x_0z^2}{1-x_0z}f_0+\left(\sum_{j} \frac{x_jcz^2}{1-x_jz}f_0\right)\\
&=-zf_0-z^2\frac{\partial f_0}{\partial z}.
\end{align*}

Then if $b$ is the coefficient of $z^{p-1}$ in $f_0$, we see that the coefficient of $z^p$ in $-zf_0$ is $-b$, and the coefficient of $z^p$ in $-z^2\frac{\partial f_0}{\partial z}$ is $-(p-1)b=b$. Therefore the coefficient of $z^p$ in $(D_1-D_0)(F_0)$ is $-b+b=0$; this means that $(D_1-D_0)(f_0)=0$. 

Then as we discussed above, the symmetry of the $f_a$ means that all the $f_a$ are annihilated by the Dunkl operators, as desired.

\end{proof}

\begin{proposition}\label{prop:linind} The $f_a$ for $a=0,\dots,n-2$ are linearly independent homogeneous polynomials of degree $p$.%, with $f_{n-1}=-\sum_{a=0}^{n-2} f_a$. 
\end{proposition} 

\begin{proof} 
We recall that $F_a=\frac{1}{1-x_az}\left(\sum_{k=0}^{p-1} \binom{c}{k} (g-1)^k\right)=\left(\sum_{k=0}^\infty x_a^kz^k\right)\left(\sum_{k=0}^{p-1} \binom{c}{k} (g-1)^k\right)$. It is then clear that $f_a$ is a homogeneous polynomial in the $x_i$ of degree $p$, since the coefficient of $z^k$ for any $k$ in $F_a$ is a homogeneous polynomial in the $x_i$ of degree $k$ for all $k$ (this follows from the fact that this is true in both multiplicands in $F_a$). 

%We also note that $\frac{\partial g}{\partial z} = g\left(\sum_a\frac{-x_az}{1-x_az}\right)$. Then

%\begin{align*}
%\sum_{a=0}^{n-1} F_a&=\left(\sum_a \frac{1}{1-x_az}\right)\left(\sum_{k=0}^{p-1} \binom{c}{k} (g-1)^k\right)\\
%&=\left(\sum_a \frac{1}{1-x_az}\right)\left(\sum_{k=0}^{p-1} \binom{c}{k} (g-1)^k\right)-n\left(\sum_{k=0}^{p-1} \binom{c}{k} (g-1)^k\right)\\
%&=\left(\sum_a\frac{x_az-1}{1-x_az}+ \frac{1}{1-x_az}\right)\left(\sum_{k=0}^{p-1} \binom{c}{k} (g-1)^k\right)\\
%&=\left(\sum_a\frac{-x_az}{1-x_az}\right)\left(\sum_{k=0}^{p-1} \binom{c}{k} (g-1)^k\right)\\
%&=\frac{z}{cg}\frac{\partial g}{\partial z}\left(\sum_{k=0}^{p-1} c\binom{c}{k} (g-1)^k\right)\\
%&=\frac{z}{c}\frac{\partial g}{\partial z}\left(\sum_{k=1}^{p-1} k\binom{c}{k} (g-1)^{k-1}\right)\\
%&=\frac{z}{c}\frac{\partial }{\partial z}\left(\sum_{k=0}^{p-1} \binom{c}{k} (g-1)^k\right)
%\end{align*}

%This equality only holds up to the $z^p$ coefficient, since we implicitly add a multiple of $(g-1)^{p-1}$ in the second-to-last step. 

%Then the coefficient of $z^p$ in this sum is $1/c$ times the coefficient of $z^{p-1}$ in $\frac{\partial }{\partial z}\left(\sum_{k=0}^{p-1} \binom{c}{k} (g-1)^k\right)$, which is $p$ times the coefficient of $z^p$ in $\sum_{k=0}^{p-1} \binom{c}{k} (g-1)^k$, which must be $0$ since we are in characteristic $p$. The coefficient of $z^p$ in this sum is also $\sum_{a=0}^{n-1} f_a$, so we have $\sum_{a=0}^{n-1} f_a=0$, and $f_{n-1}=-\sum_{a=0}^{n-2} f_a$.

Since $c$ is generic, we can write the $f_a$ as polynomials in $c$ with coefficients from the polynomial ring $A$; we can therefore consider the `constant term' of $f_a$ as a polynomial in $c$. Recall that $f_a$ is the coefficient of $z^p$ in $F_a=\left(\sum_{k=0}^\infty x_a^kz^k\right)\left(\sum_{k=0}^{p-1} \binom{c}{k} (g-1)^k\right)$. Then as polynomials it is clear that $c \mid \binom{c}{k}$ for all $k > 0$; therefore when trying to find the constant term of the coefficient of $z^p$, we can ignore the terms with $k > 0$ in the second multiplicand. The term for $k=0$ is just $1$; it is then clear that the constant term (the coefficient of $c^0$) of $f_a$ is $x_a^p$.

If $\sum_{a=0}^{n-2} \lambda_af_a=0$ for some $\lambda_a$ rational functions in $c$, we can multiply through by a least common denominator and assume the $\lambda_a$ are polynomials in $c$. We assume that not all of the $\lambda_a$ are $0$. Then we can let $e$ be the smallest nonnegative integer such that there exists an index $b$ with the coefficient of $c^e$ in $\lambda_b$ nonzero. We can divide all of the $\lambda_a$ by $c^e$, so that $\lambda_b$ for some $b$ must have nonzero constant term.

The constant term of the sum is $\sum_{a=0}^{n-2}\mu_ax_a^p$ where $\mu_a$ is the constant term of $\lambda_a$. Since the $x_a^p$ for $a=0,\dots,n-2$ are clearly linearly independent, we see that $\mu_a$ must be $0$ for $a=0,\dots,n-2$ . Then in particular the constant term $\mu_b$ of $\lambda_b$ is $0$, a contradiction. This means that our assumption that not all of the $\lambda_a$ were $0$ is false, so $\lambda_a=0$ for all $a$. 

Then since $\sum_{a=0}^{n-2} \lambda_af_a=0$ means all the $\lambda_a$ are $0$, we see that $f_a$ for $a=0,\dots,n-2$ are linearly independent as desired. 

\end{proof}


\begin{proposition}\label{prop:ci} Let $I \subseteq A$ be the ideal generated by $f_a$ for $a=0,\dots,n-2$. $A/I$ is a complete intersection. 
\end{proposition}

\begin{proof} 

We write $x$ for the vector $\langle x_0,\dots, x_{n-2} \rangle$, where the $x_i$ are taken from the rational function field in $c$ over $k$. Then we can consider $f_a$ as a function on these vectors $x$ for all $a$. For any rational function $u( c)$, we let $u(c )x=\langle u(c )x_0,\dots, u(c )x_{n-2}\rangle$. 

To show that $A/I$ is a complete intersection, we will show that if $f_a(x)=0$ for $a=0,\dots,n-2$, then $x=0$, which is an equivalent condition. 

By Proposition~\ref{prop:linind}, $f_a$ is a homogeneous polynomial in the $x_i$ of degree $p$ for all $a$. Then for any rational function $u(c )$, we see that $f_a(u(c )x)=u(c )^pf_a(x)$. In particular, if $f_a(x)=0$, then for any rational function $u(c )$ we have $f_a(u(c )x)=0$ as well. Therefore if $f_a(x) = 0$ for all $a=0,\dots,n-2$, then by choosing a particular polynomial $v(c )$ such that $v(c )x_i$ is a polynomial in $c$ for all $i=0,\dots,n-2$ (a least common denominator), we see that since $f_a(v(c )x)=0$ that we can just assume the $x_i$ are polynomials in $c$. We assume that not all of the $x_a$ are $0$. Then we can find the smallest nonnegative integer $e$ such that there exists an $b$ with the coefficient of $c^e$ in $x_b$ nonzero. Then since $f_a(\frac{1}{c^e}x)=0$ and $x_i/c^e$ is still a polynomial for any $i$, we see that we can assume that the constant term of $x_b$ for some $b$ is nonzero by dividing through by $c^e$. 

For any $a$, we can then consider $f_a(x)$ to be a polynomial in $c$. Since this is zero, we can in particular consider the constant term, which must be $0$. The constant term of the coefficient of $z^p$ is the constant term of $x_a^p$; this must be the constant term of $x_a$ raised to the $p$ power. If this is zero, then the constant term of $x_a$ must be $0$. 

Then if $f_a(x) =0$ for $a=0,\dots,n-2$, we see that the constant terms in all the $x_a$ are $0$. In particular, $x_b$ has zero constant term, a contradiction. This means that our assumption that not all of the $x_a$ were $0$ is false, so $x_a=0$ for all $a$. 

Therefore $f_0(x) = \dots = f_{n-2}(x)=0$ implies $x=0$, so $A/I$ is a complete intersection.
\end{proof}

\begin{theorem}\label{thm:main} The $f_a$ generate the ideal $J$; $A/J$ has Hilbert series $\left(\frac{1-t^p}{1-t}\right)^{n-1}$. 
\end{theorem}

\begin{proof} By Propositions~\ref{prop:linind}, \ref{prop:ci}, $A/I$ is a complete intersection with $n-1$ generators of degree $p$. It then must have Hilbert series $h_{A/I}(t)=\left(\frac{1-t^p}{1-t}\right)^{n-1}$. By Proposition~\ref{prop:ann}, the generators of $I$ are annihilated by the Dunkl operators, so $I \subseteq J$.

By Proposition 3.4 in \cite{BC1}, we see that the Hilbert series of $A/J$ is $\left(\frac{1-t^p}{1-t}\right)^{n-1}h(t^p)$ for some polynomial $h$ with nonnegative integer coefficients; since $I \subseteq J$, we see that $h_{A/I}(t) \ge h_{A/J}(t)$ coefficientwise; however by this restriction of the form of $h_{A/J}(t)$, we see that the only possible choice for $h$ is $h(t)=1$. Therefore $h_{A/I}(t)=h_{A/J}(t)$, so $I=J$ and these $n-1$ generators generate the whole ideal $J$.

\end{proof}


%something about how it's a model problem





\begin{thebibliography}{2}

\setlength{\itemsep}{-1mm}
\small

\bibitem{BC1} Martina Balagovi\'c, Harrison Chen, Representations of rational Cherednik algebras in positive characteristic, {\it J. Pure Appl. Algebra} {\bf 217} (2013), no.~4, 716--740, \arxiv{1107.0504v2}.

\bibitem{CE} Tatyana Chmutova, Pavel Etingof, On some representations of the rational Cherednik algebra, {\it Representation Theory of the American Mathematical Society} {\bf 7.24} (2003), 641--650, \arxiv{0303194v2}

\bibitem{DS} Sheela Devadas, Steven V. Sam, Representations of rational Cherednik algebras of G(m,r,n) in positive characteristic, {\it Journal of Commutative Algebra} {\bf 6.4} (2014), 525--559, \arxiv{1304.0856v2}

\bibitem{EM} Pavel Etingof, Xiaoguang Ma, Lecture notes on Cherednik algebras, \arxiv{1001.0432v4}.

\bibitem{L} Carl Lian, Representations of Cherednik algebras associated to symmetric and dihedral groups in positive characteristic, \arxiv{1207.0182v1}.

\end{thebibliography}

























\end{document}