\documentclass{amsart}
\usepackage{amsmath}
\usepackage{amsfonts}
\usepackage{amssymb,amsthm}
\usepackage{color}
\usepackage[margin=1in]{geometry}
\usepackage{graphicx}
\usepackage{verbatim}
\usepackage{hyperref}
\hypersetup{
    colorlinks,
    citecolor=blue,
    filecolor=black,
    linkcolor=black,
    urlcolor=black
}

\usepackage{todonotes}

\numberwithin{equation}{section}
\theoremstyle{definition}
\newtheorem{theorem}{Theorem}[section]
\newtheorem{proposition}[theorem]{Proposition}
\newtheorem{lemma}[theorem]{Lemma}
\newtheorem{corollary}[theorem]{Corollary}
\newtheorem{conjecture}[theorem]{Conjecture}
\newtheorem{problem}[theorem]{Problem}
\newtheorem*{remark}{Remark}
\newtheorem{definition}[equation]{Definition}
\newenvironment{defn}[1][]{%
    \begin{definition}[#1]\pushQED{\qed}}{\popQED \end{definition}}

\newcommand{\xx}[1]{\textcolor{blue}{#1}}
\newcommand{\zb}{\overline{z}}
\newcommand{\cc}[1]{\overline{#1}}
\newcommand{\m}[1]{\left|#1\right|}
\newcommand{\Res}{\operatorname{Res}}
\newcommand{\p}[1]{\left(#1\right)}
\newcommand{\rank}{\operatorname{rank}}
\newcommand{\sgn}{\operatorname{sgn}}
\newcommand{\D}{\triangle}
\newcommand{\ZZ}{\mathbb{Z}}
\newcommand{\f}[1]{\sum_{n=-\infty}^\infty #1_ne^{inx}}
\newcommand{\fn}[1]{\sum_{n=-\infty}^\infty #1e^{inx}}
\newcommand{\BB}{\mathcal{B}}
\newcommand{\M}{\mathcal{M}}
\newcommand{\mm}[1]{\m{\m{#1}}}
\newcommand{\ang}[1]{\left\langle #1 \right\rangle}
\newcommand{\h}{\mathfrak{h}}
\newcommand{\GL}{{\rm GL}}
\newcommand{\HH}{\mathcal{H}}
\newcommand{\arxiv}[1]{\href{http://arxiv.org/abs/#1}{{\tt arXiv:#1}}}
\newcommand{\Sym}{\operatorname{Sym}}
\renewcommand{\char}{\text{char}}
\newcommand{\sspan}{\text{span}}

\begin{document}
\title[Polynomial representation of type $A_{n - 1}$ rational Cherednik algebra in characteristic $p \mid n$]{The polynomial representation of the type $A_{n - 1}$ rational Cherednik algebra in characteristic $p \mid n$}
\author{Sheela Devadas}
\email[S. Devadas]{sheelad@mit.edu}
\author{Yi Sun}
\email[Y. Sun]{yisun@math.mit.edu}
\date{\today}

\begin{abstract}
We study the polynomial representation of the rational Cherednik algebra of type $A_{n-1}$ with generic parameter in characteristic $p$ for $p \mid n$. We give explicit formulas for generators for the maximal ideal, show that they cut out a complete intersection, and thus compute the Hilbert series of the irreducible quotient. Our methods are motivated by taking characteristic $p$ analogues of existing characteristic $0$ results.
\end{abstract}

\maketitle

\section{Introduction}

The present work presents a detailed study of the polynomial representation of the type $A_{n - 1}$ rational Cherednik algebra over a field of characteristic $p$ dividing $n$.  Rational Cherednik algebras were introduced by Etingof-Ginzburg in \cite{EG} as a rational degeneration of the double affine Hecke algebra dependent on two parameters $\hbar$ and $c$.  In characteristic $0$, their type $A$ representation theory has been the subject of extensive study.  We refer the reader to \cite{EM} for a survey of these results. 

In characteristic $p$ and especially in the modular case, much less is known about the representation theory of the rational Cherednik algebra.  In this paper, we consider the modular case $p \mid n$.  For $\hbar = 1$ and generic $c$, we provide a complete characterization of the irreducible quotient of the polynomial representation.  We give explicit generators for the unique maximal ideal, show that the irreducible quotient is a complete intersection, and compute its Hilbert series.

Our techniques are inspired by taking characteristic $p$ analogues of results about Cherednik algebras in characteristic $0$.  In particular, our explicit expression for generators of the maximal ideal was obtained by converting expressions containing complex residues to equivalent expressions dealing only with formal power series which may be interpreted in characteristic $p$.  While we restrict our present study to the polynomial representation in type $A$, we view this problem as a test case for this philosophy and believe it may admit wider application.

In the remainder of this introduction, we give precise statements of our results and explain how they relate to other recent work.

\subsection{The rational Cherednik algebra in positive characteristic}

Throughout this paper, we work over an algebraically closed field $k$ of characteristic $p > 0$ and fix $n$ so that $p \mid n$.  Let $S_n$ denote the symmetric group on $n$ elements, $V = k^n$ its permutation representation, and $s_{ij} \in S_n$ the transposition permuting $i$ and $j$.  Fix a basis $y_0,\ldots,y_{n-1}$ for $V$ and a dual basis $x_0, \ldots, x_{n-1}$ for $V^*$.  Let $\h$ and $\h^*$ be the dual $(n - 1)$-dimensional $S_n$-representations which are subrepresentation and quotient of $V$ and $V^*$, respectively given by
\[
\h = \sspan\{y_i - y_j \mid i \neq j\} \text{ and } \h^* = V^*/(x_0 + \cdots + x_{n - 1}).
\]
The action of $S_n$ on $\h$ and $\h^*$ is given explicitly by natural permutation of basis vectors.

Fix constants $\hbar$ and $c$ in $k$.  Denoting the tensor algebra of $\h \oplus \h^*$ by $T(\h \oplus \h^*)$, the \textit{type $A_{n - 1}$ rational Cherednik algebra} $\HH_{\hbar, c}(\h)$ is the quotient of $k[S_n] \ltimes T(\h \oplus \h^*)$ by the relations
\begin{align*}
[x_i,x_j]=0, \quad [y_i,y_j] = 0, \quad [y_i,x_j] = \hbar\langle y_i,x_j\rangle - \sum_{k=0}^{n-1}\sum_{l=0}^{n-1} c\langle y_i,x_k-x_l \rangle\langle y_k-y_l,x_j\rangle s_{kl}
\end{align*}
for all $0 \le i,j \le n-1$.  There is a $\ZZ$-grading on $\HH_{\hbar,c}(\h)$ given by setting $\deg x=1$ for $x \in \h^*$, $\deg y = -1$ for $y \in \h$, and $\deg g=0$ for $g \in k[S_n]$.  In addition, $\HH_{\hbar, c}(\h)$ admits a PBW decomposition 
\[
\HH_{\hbar,c}(\h) = \Sym(\h) \otimes_k k[S_n] \otimes_k \Sym(\h^*).
\]
For any $\alpha \ne 0$, $\HH_{\hbar,c}(\h)$ and $\HH_{\alpha\hbar,\alpha c}(\h)$ are isomorphic as algebras, so only the cases $\hbar = 0$ or $\hbar = 1$ need be considered.  In this paper, we restrict our attention to $\hbar = 1$.

\subsection{Polynomial representation of the rational Cherednik algebra}

The rational Cherednik algebra $\HH_{\hbar, c}(\h)$ admits a $\ZZ_{\geq 0}$-graded representation on the polynomial ring $A = \Sym(\h^*)$, known as the \textit{polynomial representation}. The actions of $\Sym(\h^*)$ and $k[S_n]$ on $A$ are by left multiplication and the $S_n$ action on $\h^*$, respectively.  The action of $\Sym(\h)$ is implemented by letting $y \in \h$ act via the \textit{Dunkl operator} $D_y$ defined by
\begin{align*}
D_y(f) = \hbar \partial_y f  - \sum_{k=0}^{n-1} \sum_{l=0}^{n-1} c  \langle y, x_k-x_l \rangle \frac{(1-s_{kl}) f}{x_k - x_l},
\end{align*}
where we note that $\frac{(1-s_{kl}) f}{x_k - x_l}$ is a polynomial for $f \in A$. Explicitly, for 
\[
D_i = ,
\]
we have $D_{y_i - y_j} = D_i - D_j$. 

\todo[inline]{Fill this in, $D_{y_i}$ is not well-defined because $y_i$ is not in $\h$}


\subsection{Maximal ideal and irreducible quotient of the polynomial representation}

As described in \cite[Section 2.5]{BC1}, there is a contravariant form 
\[
\beta_c: \Sym(\h^*) \otimes \Sym(\h) \to k
\]
defined by setting $\beta_c(1, 1) = 1$ and imposing for all $x \in \h^*, y \in \h, f \in \Sym(\h^*), g \in \Sym(\h)$ that
\[
\beta_c(xf,g)=\beta_c(f,xg) \qquad \text{ and } \qquad \beta_c(f,yg) = \beta_c(yf,g).
\]
Then the polynomial representation $\Sym(\h^*)$ has unique maximal graded proper submodule $J = \ker(\beta_c)$, and the quotient $L = A/J$ is an irreducible representation of $\HH_{\hbar,c}(\h)$.  It inherits a $\ZZ_{\geq 0}$-grading from $A$, and we recall that its Hilbert series is 
\[
h_L(t) = \sum_{j \geq 0} \dim L_j t^j,
\]
where $L_j$ is the degree $j$ subspace of $L$. 

\subsection{Statement of the main result}

For a formal Laurent series $r(z)$, we denote by $[z^l] r(z)$ the coefficient of $z^l$ in $r(z)$.  For $i = 0, \ldots, n - 2$, define the formal power series
\[
F_i(z)=\frac{1}{1-x_iz} \left( \sum_{k=0}^{p-1} \binom{c}{k}(-1+\prod_{j=0}^{n-1} (1-x_jz))^k\right),
\]
and consider the coefficients $f_i = [z^p] F_i(z)$.

\newtheorem*{thm:main}{Theorem \ref{thm:main}} \begin{thm:main}
For generic $c$, the $f_i$ are linearly independent and generate the maximal ideal $J$ of the polynomial representation for $\HH_{1, c}(\h)$, and the irreducible quotient $L = A/J$ is a complete intersection with Hilbert series 
\[
h_L(t) = \left(\frac{1-t^p}{1-t}\right)^{n-1}.
\]
\end{thm:main}
\begin{remark}
In Theorem \ref{thm:main}, by generic $c$ we mean $c$ avoiding countably many values.
\end{remark}

\subsection{Connections to previous work}

Our study is motivated by previous work on the representation theory of the type $A$ rational Cherednik algebra in both characteristic $0$ and $p$.  The type $A$ non-modular case $p \gg n$ was studied in \cite{BFG}, and some properties of the maximal ideal of the polynomial representation were given in both modular and non-modular cases in \cite{BC1}.  In the modular case $p \mid n$, for $p = 2$, the polynomial representation associated to the $n$-dimensional permutation representation was studied.
\begin{theorem}[{\cite[Theorem 5.1]{L}}] \label{thm:lian}
The irreducible quotient of the polynomial representation associated to the $n$-dimensional reflection representation is a complete intersection with Hilbert series
\[
h(t) = (1 + t)^n (1 + t^2).
\]
The corresponding maximal ideal is generated by $n - 1$ elements of degree $2$ and one element of degree $4$. 
\end{theorem} 
It was further conjectured by Lian in \cite[Conjecture 5.2]{L} that for all $p$ the corresponding irreducible is a complete intersection with maximal ideal having $n - 1$ generators in degree $p$ and a single generator in degree $p^2$.  Our results are consistent with the restriction of Lian's conjecture to the case when $\h$ is restricted to the $(n - 1)$-dimensional quotient.  It would be interesting to extend our work to prove Lian's conjecture in full.  For general $p \mid n$, a subideal of the maximal ideal was computed in \cite[Proposition 6.1]{DS}.

In characteristic $0$, our results parallel the explicit decomposition of the polynomial representation of the type $A$ rational Cherednik algebra given in \cite{BEG, CE}.  There, the polynomial representation is irreducible unless $c = \frac{r}{n}$ for some integer $r$, and an explicit set of generators of the maximal ideal is known.
\begin{proposition}[{\cite[Proposition 3.1]{CE}}] \label{prop:ce}
If $\char(k) = 0$ and $c = \frac{r}{n}$, the maximal ideal $J \subset A$ of the polynomial representation $A$ of $\HH_{\hbar,c}(\h)$ is generated by
\[
\Res_\infty\left[\frac{dz}{z-x_j} \prod_{i=0}^{n-1} (z-x_i)^c\right] \text{ for $j=0,\dots,n-2$}.
\]
\end{proposition}
We interpret the characteristic $p$ analogue of Proposition \ref{prop:ce} to mean that if $r = p$ and $p \mid n$, then taking $c = \frac{p}{n}$ should correspond to taking generic $c$.  While this substitution is of course invalid, Proposition \ref{prop:ce} may be interpreted as a statement about certain formal power series.  By using a power series version of this construction of generators which  makes sense in characteristic $p$, we are able to mimic the arguments of \cite{BEG, CE} to show that they cut out a complete intersection and generate the entire ideal.  We believe that the philosophy of taking characteristic $p$ analogues of characteristic $0$ results for the rational Cherednik algebra should apply more generally and hope to explore this further in future work.

\subsection{Outline of the paper}

The remainder of this paper is organized as follows.  In Section 2, we check that the generators $f_i$ are linearly independent singular vectors.  In Section 3, we show that they cut out a complete intersection.  In Section 4, we put these facts together to conclude Theorem \ref{thm:main}.

\subsection{Acknowledgements} 

The authors thank P. Etingof for suggesting the problem and for helpful discussions.  Some exploratory computations were done using Sage.  S.~D. was supported by the MIT Undergraduate Research Opportunities Program (UROP). Y.~S. was supported by a NSF Graduate Research Fellowship (NSF Grant \#1122374).

\todo[inline]{Ask Pavel for his grant number}

\section{An explicit construction of singular vectors}

We let $g(z)=\prod_{j=0}^{n-1} (1-x_jz)$ and $F(z)$ be the formal power series in $z$ defined by
\[
F(z) =\left(\sum_{k=0}^{p-1} \binom{c}{k} (g-1)^k\right).
\]
where $\binom{c}{k}=\frac{c(c-1)\dots(c-k+1)}{k!}$. 
Let $F_i(z)$ for $i=0,\dots,n-1$ be $\frac{F(z)}{1-x_iz}$ and $f_i=[z^p]F_i(z)$. The $f_i$ will be the generators of the maximal ideal $J$. 

Because of the definition of the contravariant form $\beta_c$, showing that an element $f$ of $A$ is in the kernel of $\beta_c$ is equivalent to showing that the Dunkl operators corresponding to the basis elements of $\h$ annihilate $f$.

\begin{proposition}\label{prop:ann} The elements $f_i$ for $i=0,\dots,n-1$ are annihilated by $D_i-D_j$ for all $i \ne j$.
\end{proposition}

\begin{proof}

We see that for all $i$, $f_i$ is symmetric in the $x_j$ for $j \ne i$. It therefore suffices to consider the action of the Dunkl operators on $f_0$. We note that the Dunkl operators $D_i-D_j$ for all $i \ne j$ are spanned by $D_i-D_0$ for $0<i \le n-1$; then because $f_0$ is symmetric in the $x_i$ with $i > 0$, we see that it suffices to consider the action of $D_1-D_0$ on $f_0$ to show that the $f_i$ are annihilated by $D_i \ne D_j$ for all $i \ne j$. 

Since $x_{n-1}=-x_0-\dots-x_{n-2}$, we see that
\[
\frac{\partial g}{\partial x_i}=-\frac{zg}{1-x_iz}+\frac{zg}{1-x_{n-1}z} \text{ for all $0 \le i < n-1$}.
\]
We recall that $F(z)$ is symmetric and $F_a(z)=\frac{F(z)}{1-x_az}$ for all $a$. Then we see that for all $0 \le i < n-1$ we have 
\begin{align*}
\frac{\partial F}{\partial x_i}%&=\frac{\partial}{\partial x_i}\left(\sum_{k=0}^{p-1} \binom{c}{k} (g-1)^k\right)\\
&=\left(\sum_{k=1}^{p-1}k\binom{c}{k}(g-1)^{k-1}\right)\frac{\partial g}{\partial x_i}\\
&=\left(-\frac{z}{1-x_iz}+\frac{z}{1-x_{n-1}z}\right)\left(\sum_{k=1}^{p-1}k\binom{c}{k}(g-1)^{k-1}\right)g\\
%&=\left(-\frac{z}{1-x_iz}+\frac{z}{1-x_{n-1}z}\right)\left(\sum_{k=0}^{p-2}(k+1)\binom{c}{k+1}(g-1)^{k}\right)g\\
%&=\left(-\frac{z}{1-x_iz}+\frac{z}{1-x_{n-1}z}\right)\left(\sum_{k=0}^{p-2}c\binom{c-1}{k}(g-1)^{k}\right)(g-1+1)\\
&=\left(-\frac{z}{1-x_iz}+\frac{z}{1-x_{n-1}z}\right)\left(\left(\sum_{k=0}^{p-2}c\binom{c-1}{k}(g-1)^{k}\right)+\left(\sum_{k=0}^{p-2}c\binom{c-1}{k}(g-1)^{k+1}\right)\right)\\
%&=\left(-\frac{z}{1-x_iz}+\frac{z}{1-x_{n-1}z}\right)\left(\left(\sum_{k=0}^{p-2}c\binom{c-1}{k}(g-1)^{k}\right)+\left(\sum_{k=1}^{p-1}c\binom{c-1}{k-1}(g-1)^{k}\right)\right)\\
%&=\left(-\frac{z}{1-x_iz}+\frac{z}{1-x_{n-1}z}\right)\left(-c\binom{c-1}{p-1}(g-1)^{p-1}+\sum_{k=0}^{p-1}c\binom{c}{k}(g-1)^{k}\right)\\
&=\left(-\frac{zc}{1-x_iz}+\frac{zc}{1-x_{n-1}z}\right)\left(-\binom{c-1}{p-1}(g-1)^{p-1}+F(z)\right).\\
\end{align*}
We define
\[
G_i(z)=\left(-\frac{zc}{1-x_iz}+\frac{zc}{1-x_{n-1}z}\right)\left(-\binom{c-1}{p-1}(g-1)^{p-1}\right).
\]
Then we see that 
\[
\frac{\partial F}{\partial x_i}=G_i(z)+\left(-\frac{zc}{1-x_iz}+\frac{zc}{1-x_{n-1}z}\right)F(z).
\]
We can now calculate $\frac{\partial F_0}{\partial x_1},\frac{\partial F_0}{\partial x_0}$. 
\begin{align*}
\frac{\partial F_0}{\partial x_1}%&=\frac{\partial}{\partial x_1}\left(\frac{1}{1-x_0z}\left(\sum_{k=0}^{p-1} \binom{c}{k} (g-1)^k\right)\right)\\
&=\frac{1}{1-x_0z}\frac{\partial F}{\partial x_1}\\
&=\frac{1}{1-x_0z}\left(-\frac{zc}{1-x_1z}+\frac{zc}{1-x_{n-1}z}\right)F(z)+\frac{G_1(z)}{1-x_0z}\\
&=\left(-\frac{zc}{1-x_1z}+\frac{zc}{1-x_{n-1}z}\right)F_0(z)+\frac{G_1(z)}{1-x_0z}.\\
\frac{\partial F_0}{\partial x_0}%&=\frac{\partial}{\partial x_1}\left(\frac{1}{1-x_0z}\left(\sum_{k=0}^{p-1} \binom{c}{k} (g-1)^k\right)\right)\\
&=\frac{z}{(1-x_0z)^2}F(z)+\frac{1}{1-x_0z}\frac{\partial F}{\partial x_0}\\
&=\frac{z}{1-x_0z}F_0(z)+\frac{1}{1-x_0z}\left(-\frac{zc}{1-x_0z}+\frac{zc}{1-x_{n-1}z}\right)F(z)+\frac{G_0(z)}{1-x_0z}\\
%&=\frac{z}{1-x_0z}F_0(z)+\left(-\frac{zc}{1-x_0z}+\frac{zc}{1-x_{n-1}z}\right)F_0(z)+\frac{G_0(z)}{1-x_0z}\\
&=\left(\frac{z(1-c)}{1-x_0z}+\frac{zc}{1-x_{n-1}z}\right)F_0(z)+\frac{G_0(z)}{1-x_0z}.\\
\end{align*}
We note that $F_0(z)$ is invariant under $s_{ij}$ where $0 < i,j$. Therefore for transpositions we need only consider transpositions of the form $s_{0i}$ for $0 < i \le n-1$. We see that 
\begin{align*}
\frac{1-s_{0i}}{x_0-x_i}(F_0)%&=\frac{1}{x_0-x_i}\left(\frac{F(z)}{1-x_0z}-\frac{F(z)}{1-x_iz}\right)\\
&=\frac{1}{x_0-x_i}\left(\frac{1}{1-x_0z}-\frac{1}{1-x_iz}\right)F(z)\\
%&=\frac{1}{x_0-x_i}\left(\frac{(1-x_iz)-(1-x_0z)}{(1-x_0z)(1-x_iz)}\right)F(z)\\
%&=\frac{x_0z-x_iz}{(1-x_0z)(1-x_iz)(x_0-x_i)}F(z)\\
&=\frac{z}{(1-x_iz)(1-x_0z)}F(z)\\
&=\frac{z}{1-x_iz}F_0(z).\\
\end{align*}
We recall that we need only consider the action of $D_1-D_0$ on $F_0(z)$. We consider $D_0F_0, D_1F_0$ separately first. We see that 
\[
D_0=\left(\frac{\partial}{\partial x_0}-c\sum_{j > 0} \frac{1-s_{0j}}{x_0-x_j}\right), D_1=\left(\frac{\partial}{\partial x_1}-c \frac{1-s_{01}}{x_1-x_0}\right)
\]
since $F_0$ is invariant under $s_{ij}$ where $0 < i,j$. We now compute
\begin{align*}
D_0F_0&=\left(\frac{\partial}{\partial x_0}-c\sum_{j > 0} \frac{1-s_{0j}}{x_0-x_j}\right)(F_0)\\
&=\frac{\partial F_0}{\partial x_0}-c\sum_{j > 0} \frac{1-s_{0j}}{x_0-x_j}(F_0)\\
&=\frac{G_0(z)}{1-x_0z}+\left(\frac{z(1-c)}{1-x_0z}+\frac{zc}{1-x_{n-1}z}\right)F_0(z)-\sum_{j > 0} \frac{zc}{1-x_jz}F_0(z).\\
D_1F_0&=\left(\frac{\partial}{\partial x_1}-c \frac{1-s_{01}}{x_1-x_0}\right)(F_0)\\
&=\frac{\partial F_0}{\partial x_1}+c \frac{1-s_{01}}{x_0-x_1}(F_0)\\
&=\frac{G_1(z)}{1-x_0z}+\left(-\frac{zc}{1-x_1z}+\frac{zc}{1-x_{n-1}z}\right)F_0(z)+\frac{zc}{1-x_1z}F_0(z)\\
&=\frac{G_1(z)}{1-x_0z}+\frac{zc}{1-x_{n-1}z}F_0(z).\\
\end{align*}
It is then easy to see that 
\[
(D_1-D_0)(F_0)=\frac{G_1-G_0}{1-x_0z}+\frac{z(c-1)}{1-x_0z}F_0+\sum_{j>0} \frac{zc}{1-x_jz}F_0.
\]
In order to show that the $p$th coefficient in this formal power series is $0$, we must consider $\frac{\partial F_0}{\partial z}$. 

We see easily that 
\[
\frac{\partial g}{\partial z} = g(z) \sum_j \frac{-x_j}{1-x_jz}.
\]
We now consider $\frac{\partial F}{\partial z}$. We compute
\begin{align*}
\frac{\partial F}{\partial z}%&=\frac{\partial}{\partial z}\left(\sum_{k=0}^{p-1} \binom{c}{k} (g-1)^k\right)\\
&=\left(\sum_{k=1}^{p-1}k\binom{c}{k}(g-1)^{k-1}\right)\frac{\partial g}{\partial z}\\
%&=\left(\sum_j \frac{-x_j}{1-x_jz}\right)\left(\sum_{k=1}^{p-1}k\binom{c}{k}(g-1)^{k-1}\right)g(z)\\
%&=\left(\sum_j \frac{-x_j}{1-x_jz}\right)\left(\sum_{k=0}^{p-2}(k+1)\binom{c}{k+1}(g-1)^{k}\right)g(z)\\
&=\left(\sum_j \frac{-x_j}{1-x_jz}\right)\left(\sum_{k=0}^{p-2}c\binom{c-1}{k}(g-1)^{k}\right)(g-1+1)\\
%&=\left(\sum_j \frac{-x_j}{1-x_jz}\right)\left(\left(\sum_{k=0}^{p-2}c\binom{c-1}{k}(g-1)^{k}\right)+\left(\sum_{k=0}^{p-2}c\binom{c-1}{k}(g-1)^{k+1}\right)\right)\\
&=\left(\sum_j \frac{-x_j}{1-x_jz}\right)\left(\left(\sum_{k=0}^{p-2}c\binom{c-1}{k}(g-1)^{k}\right)+\left(\sum_{k=1}^{p-1}c\binom{c-1}{k-1}(g-1)^{k}\right)\right)\\
&=\left(\sum_j \frac{-x_j}{1-x_jz}\right)\left(-c\binom{c-1}{p-1}(g-1)^{p-1}+\sum_{k=0}^{p-1}c\binom{c}{k}(g-1)^{k}\right)\\
&=\left(\sum_j \frac{-cx_j}{1-x_jz}\right)F(z)+\left(\sum_j \frac{-x_j}{1-x_jz}\right)\left(-c\binom{c-1}{p-1}(g-1)^{p-1}\right).
\end{align*}
We define 
\[
V(z)=\left(\sum_j \frac{-x_j}{1-x_jz}\right)\left(-c\binom{c-1}{p-1}(g-1)^{p-1}\right).
\]
Then we see that 
\[
\frac{\partial F}{\partial z}=V(z)+\left(\sum_j \frac{-cx_j}{1-x_jz}\right)F(z).
\]
From this it follows that
\begin{align*}
\frac{\partial F_0}{\partial z} &= \frac{\partial}{\partial z}\left(\frac{F(z)}{1-x_0z}\right)\\
&=\frac{1}{1-x_0z} \frac{\partial F}{\partial z}+\frac{x_0}{1-x_0z} F(z)\\
&=\frac{V(z)}{1-x_0z}+\frac{1}{1-x_0z}\left(\sum_j \frac{-cx_j}{1-x_jz}\right)F(z)+\frac{x_0}{(1-x_0z)^2} F(z)\\
&=\frac{V(z)}{1-x_0z}+\left(\sum_j \frac{-cx_j}{1-x_jz}\right)F_0(z)+\frac{x_0}{1-x_0z} F_0(z).\\
\end{align*}
We now again consider $(D_1-D_0)(F_0)$. Recall that $n \equiv 0 \bmod p$, so in particular we can add $n$ times any multiple of $F_0$ since that is $0$ in characteristic $p$. 
\begin{align*}
(D_1-D_0)(F_0)&=\frac{G_1(z)-G_0(z)}{1-x_0z}+\frac{z(c-1)}{1-x_0z}F_0+\sum_{j>0} \frac{zc}{1-x_jz}F_0\\
%&=\frac{G_1(z)-G_0(z)}{1-x_0z}-\frac{z}{1-x_0z}F_0+\sum_{j} \frac{zc}{1-x_jz}F_0\\
%&=\frac{G_1(z)-G_0(z)}{1-x_0z}-\frac{z}{1-x_0z}F_0+\left(\sum_{j} \frac{zc}{1-x_jz}F_0\right)-nzcF_0\\
&=\frac{G_1(z)-G_0(z)}{1-x_0z}-zF_0+zF_0-\frac{z}{1-x_0z}F_0+\left(\sum_{j} -zcF_0+\frac{zc}{1-x_jz}F_0\right)\\
%&=\frac{G_1(z)-G_0(z)}{1-x_0z}-zF_0+\frac{z-x_0z^2}{1-x_0z}F_0-\frac{z}{1-x_0z}F_0+\left(\sum_{j} \frac{-zc+x_jcz^2}{1-x_jz}F_0+\frac{zc}{1-x_jz}F_0\right)\\
&=\frac{G_1(z)-G_0(z)}{1-x_0z}-zF_0+\frac{-x_0z^2}{1-x_0z}F_0+\left(\sum_{j} \frac{x_jcz^2}{1-x_jz}F_0\right)\\
%&=\frac{G_1(z)-G_0(z)}{1-x_0z}-zF_0-z^2\left(\frac{x_0}{1-x_0z}F_0+\left(\sum_{j}- \frac{x_jc}{1-x_jz}F_0\right)\right)\\
&=\frac{G_1(z)-G_0(z)}{1-x_0z}-zF_0-z^2\frac{\partial F_0}{\partial z}+z^2\frac{V(z)}{1-x_0z}\\
&=\frac{G_1(z)-G_0(z)+z^2V(z)}{1-x_0z}-zF_0-z^2\frac{\partial F_0}{\partial z}.\\
\end{align*}
We consider $z^2V(z)$ and $G_i(z)$ for $i=0,1$. We note that $(g-1)^{p-1}$ divides $V(z)$ and $z(g-1)^{p-1}$ divides $G_i(z)$. Since the sum $\sum_{a=0}^{n-1} x_a$ is $0$, we see that $z^2 \mid g(z)-1$; therefore $z^{2p-2} \mid (g-1)^{p-1}$. Therefore $z^{p+1} \mid z^{2p-1} \mid G_i(z)$ and $z^{p+1} \mid z^{2p} \mid z^2V(z)$. (We note that $z^p \mid z^{2p-2} \mid V(z)$; this will be relevant later.)

Then we see that for $0 \le \ell \le p$ we have
\[
[z^\ell](z^2V(z))=[z^\ell]G_0(z)=[z^\ell]G_1(z)=0.
\]
Then we also have
\[
[z^\ell](G_1(z)-G_0(z)+z^2V(z))=0.
\]
Then $[z^p]\left(\frac{G_1-G_0+z^2V}{1-x_0z}\right)$ is $0$ since it is a linear combination of $[z^\ell](G_1(z)-G_0(z)+z^2V(z))$ for $0 \le \ell \le p$, so 
\[
[z^p]\left((D_1-D_0)(F_0)\right)=[z^p]\left(-zF_0-z^2\frac{\partial F_0}{\partial z}\right).
\]
Let $b=[z^{p-1}](F_0)$. We see that $[z^p](-zF_0)=-b$. Then $[z^{p-2}]\left(\frac{\partial F_0}{\partial z}\right)=(p-1)b=-b$, so $[z^p]\left(-z^2\frac{\partial F_0}{\partial z}\right)=b$. Therefore $[z^p]\left(-zF_0-z^2\frac{\partial F_0}{\partial z}\right)=-b+b=0$, so 
\[
[z^p]((D_1-D_0)(F_0))=(D_1-D_0)([z^p]F_0)=(D_1-D_0)f_0=0.
\]
By the symmetry of the $F_i$, this is the only Dunkl operator we need consider; it is clear that all the $f_i$ are killed by the Dunkl operators.
\end{proof}

\begin{proposition}\label{prop:linind} The $f_i$ for $i=0,\dots,n-2$ are linearly independent homogeneous polynomials of degree $p$.
\end{proposition} 

\begin{proof} 
We recall that $F_i(z)=\frac{1}{1-x_iz}\left(\sum_{k=0}^{p-1} \binom{c}{k} (g-1)^k\right)=\left(\sum_{k=0}^\infty x_i^kz^k\right)\left(\sum_{k=0}^{p-1} \binom{c}{k} (g-1)^k\right)$. It is then clear that $f_i$ is a homogeneous polynomial in the $x_j$ of degree $p$, since $[z^k]F_i(z)$ is a homogeneous polynomial in the $x_j$ of degree $k$ for all $k$ (this follows from the fact that this is true in both multiplicands in $F_i$). 

Assume $\sum_{i=0}^{n-2} \lambda_if_i=0$ for some $\lambda_i \in k$. Then in particular this must be true when we set $x_j=1$ and $x_i=0$ for $i \ne j,i < n-1$. (We let $x_{n-1}=-1$ so that $\sum x_i=0$.) Note that if this is the case that $g(z)=(1-z)(1+z)=1-z^2$, so 
\[
F_j(z)=\left(\sum_{k=0}^\infty z^k\right)\left(\sum_{k=0}^{p-1} \binom{c}{k} (-z^2)^k\right)
\]
and 
\[
F_i(z)=\left(\sum_{k=0}^\infty 0^kz^k\right)\left(\sum_{k=0}^{p-1} \binom{c}{k} (-z^2)^k\right)=\sum_{k=0}^{p-1} \binom{c}{k} (-z^2)^k
\]
for $n-1>i \ne j$.

We first consider the case $p=2$. In this case $[z^2]F_j(z)=1-c$ and $[z^2]F_i(z)=-c$. By setting $x_j=1,x_{n-1}=-1$ and $x_i=0$ for $i \ne j, i \ne n-1$ for all $j=0,\dots,n-1$ in succession, we find that 
\[
\lambda_j-c\sum_{i=0}^{n-2} \lambda_i = 0 \text{ for all }j.
\]
Therefore for all $j$, $\lambda_j=c\sum_{i=0}^{n-2} \lambda_i$. Then in particular all of the $\lambda_i$ are equal to some $\lambda \in k$ with $(1-c(n-1))\lambda = 0$. Since $c$ is generic, we can assume $(1-c(n-1)) \ne 0$, so then $\lambda=0$. Therefore $\lambda_j=0$ for all $j$; then the $f_i$ for $i=0,\dots,n-2$ are linearly independent as desired.


We now suppose $p>2$. Then we have $[z^p]F_j(z)=f_j=\sum_{k=0}^{(p-1)/2} (-1)^k\binom{c}{k}=\binom{c-1}{(p-1)/2}$ and $[z^p]F_i(z)=f_i=0$ for $n-1>i \ne j$ (since $p > 2$). Since $c$ is generic, we can assume $\binom{c-1}{(p-1)/2} \ne 0$. Then since $\sum_{i=0}^{n-2} \lambda_if_i=0$, we see that $\lambda_j\binom{c-1}{(p-1)/2}=0$; then we must have $\lambda_j=0$.

By setting $x_j=1,x_{n-1}=-1$ and $x_i=0$ for $i \ne j, i \ne n-1$ for all $j=0,\dots,n-1$ in succession, we find that $\lambda_j=0$ for all $j$. Therefore the $f_i$ for $i=0,\dots,n-2$ are linearly independent as desired.
\end{proof}

\section{Complete interesction properties}

\begin{proposition}\label{prop:vand}Let $y_0, \ldots, y_{r - 1}$ be distinct elements of $k$.  Then the polynomials $b_i(z) = \prod_{j \neq i} (1 - y_j z)$ are linearly independent.
\end{proposition}
\begin{proof} 
Suppose that for some $\lambda_i \in k$ we have $\sum_i \lambda_ib_i(z)=0$.  Then $\left(\prod_j (1-y_jz)\right)\left(\sum_i \frac{\lambda_i}{1-y_iz}\right)$ must be $0$ in the ring of formal power series. Since the ring of power series is an integral domain, $\sum_i \frac{\lambda_i}{1-y_iz}=0$. In particular for $\ell=0,\dots,r-1$ we have $\sum_i \lambda_iy_i^\ell=0$.  However, the vectors $(1,y_i,\dots,y_i^{r-1})$ are linearly independent by the Vandermonde determinant, so $\lambda_i=0$ for all $i$, meaning that $b_i(z)$ are linearly independent.
\end{proof}

\begin{proposition}\label{prop:ci} Let $I \subseteq A$ be the ideal generated by $f_i$ for $i=0,\dots,n-2$.  Then $A/I$ is a complete intersection. 
\end{proposition}

\begin{proof} 

We will show that if $x_0,\dots,x_{n-1} \in k$ satisfy $f_i(x_0,\dots,x_{n-1})=0$ for all $i$, then $x_0=\dots=x_{n-1}=0$. 

We suppose that the $x_i$ take the distinct values $\{y_0,\dots,y_{r-1}\}$ with multiplicity $m_i>0$. Then $g(z)=(1-y_iz)^{m_i}$. Recall that
\[
F(z)=\sum_{k=0}^{p-1} \binom{c}{k}(g-1)^k.
\]
Then we assume that $[z^p]\left(\frac{1}{1-y_iz}F(z)\right)=0$ for all $i=0,\dots,r-1$. We wish to show that $r=1$ and $y_0=0$. 

We will assume that $\prod_i y_i \ne 0$ in order to derive a contradiction and show that one of the $y_i$ is $0$. Since $\sum_{i=0}^{n-1} x_i=0$ we also have $\sum_{i=0}^{r-1} m_iy_i=0$; this enables us to compute the derivative $F'(z)$, which is 
\[
F'(z)=V(z)+\left(\sum_j \frac{-cx_j}{1-x_jz}F(z)\right)=V(z)+\left(\sum_j \frac{-cm_jy_j}{1-y_jz}F(z)\right),
\]
where $[z^\ell]V(z)=0$ for $0 \le \ell \le p$. 

We will use a `residues by parts' argument; using the fact that the residue of an exact differential is zero, we will relate the coefficients of a power series to the coefficients of the derivative of that power series and derive a contradiction. 
Let
\[
a(z)=\frac{1}{\prod_{i=0}^{r-1}(1-y_iz)}F(z).
\]
We then see that
\begin{align*}
a'(z)&=\frac{d}{dz}\left(\frac{1}{\prod_{i=0}^{r-1}(1-y_iz)}F(z)\right)\\
&=\frac{1}{\prod_{i=0}^{r-1}(1-y_iz)}F'(z)+\sum_{j=0}^{r-1}\frac{y_j}{(1-y_jz)\prod_{i=0}^{r-1}(1-y_iz)}F(z)\\
&=\frac{1}{\prod_{i=0}^{r-1}(1-y_iz)}\left(V(z)+\left(\sum_j \frac{-cm_jy_j}{1-y_jz}F(z)\right)\right)+\sum_{j=0}^{r-1}\frac{y_j}{1-y_jz}a(z)\\
&=\frac{V(z)}{\prod_{i=0}^{r-1}(1-y_iz)}+\left(\sum_j \frac{-cm_jy_j}{1-y_jz}a(z)\right)+\sum_{j=0}^{r-1}\frac{y_j}{1-y_jz}a(z)\\
&=\frac{V(z)}{\prod_{i=0}^{r-1}(1-y_iz)}-\left(\sum_j \frac{(m_jc-1)y_j}{1-y_jz}a(z)\right).
\end{align*}
Since $[z^\ell]V(z)=0$ for $0 \le \ell \le p$ we see that $\frac{V(z)}{\prod_{i=0}^{r-1}(1-y_iz)}=z^pb(z)$ for some power series $b(z)$. Therefore
\[
a'(z)=z^pb(z)-\sum_j \frac{(m_jc-1)y_j}{1-y_jz}a(z)
\]
for some power series $b(z)$.

Define $b_i(z)=\prod_{j \ne i} (1-y_jz)$ for $i=0,\dots,r-1$; then by Proposition~\ref{prop:vand} we see that the polynomials $b_i(z)$ are linearly independent. 

Note that for any $i$, $a(z)b_i(z)=\frac{1}{1-y_iz}F(z)$. Then in particular, we have 
\[
[z^p]a(z)b_i(z)=[z^p]\frac{1}{1-y_iz}F(z)=0
\]
Therefore for any $\lambda_i \in k$, we have
\begin{equation}\label{eq:lincomb}
\sum_{i=0}^{r-1} \lambda_i[z^p]a(z)b_i(z)=[z^p]\left( a(z)\sum_{i=0}^{r-1}\lambda_ib_i(z)\right)=0.
\end{equation}
Recall that the $b_i(z)$ are linearly independent and have degree at most $r-1$. Therefore for $0 \le k \le r-1$ we can choose $\{\lambda_i^k\}$ so that 
\[
\sum_i\lambda_i^kb_i(z)=z^k.
\]
Choosing these $\lambda_i^k$ in (\ref{eq:lincomb}), we see that $[z^p](a(z)z^k)=[z^{p-k}]a(z)=0$ for $k=0,\dots,r-1$. 

We will now derive from this that $[z^0]a(z)=0$, a contradiction. If $r > p$ then this has already been established, so it suffices to assume $r \le p$. 

For any integer $\ell$, define the Laurent polynomial 
\[
h_\ell(z)=z^{-\ell-r}\prod_i(1-y_iz).
\]
Notice that
\begin{align*}
\frac{d}{dz}( h_\ell(z) a(z)) &= -(l + r)z^{-1} h_\ell(z) a(z) - \sum_i \frac{y_i}{1 - y_iz} h_\ell(z) a(z) + h_\ell(z) a'(z) \\
&=  -(\ell + r)z^{-1} h_\ell(z) a(z) - \sum_i \frac{y_i}{1 - y_iz} h_\ell(z) a(z) - h_\ell(z) \sum_j \frac{(m_j c - 1)y_i}{1 - y_ijz} a(z)  + z^ph_\ell(z) b(z)\\
&= -(\ell + r)z^{-1} h_\ell(z) a(z) - \sum_i \frac{y_im_i c}{1 - y_iz} h_\ell(z) a(z) + z^ph_\ell(z) b(z)\\
&= -\left((\ell + r) z^{-1} + \sum_i \frac{y_im_i c}{1 - y_iz}\right) h_\ell(z) a(z) + z^ph_\ell(z) b(z).
\end{align*}
We note that 
\[
-\left((\ell + r) z^{-1} + \sum_i \frac{y_im_i c}{1 - y_iz}\right) h_\ell(z)
\]
is a Laurent polynomial with lowest degree term 
\[
- z^{-l-r-1} \left(\ell + r + \sum_i y_i m_i c\right) = - z^{-\ell-r-1}(\ell+ r)
\]
and highest degree term 
\[
- z^{-\ell - 1} (-1)^r  \prod_i y_i\left((\ell + r) - \sum_i m_i c\right).
\]
The lowest degree term of the formal power series $z^ph_\ell(z)b(z)$ has degree at least $p-\ell-r$. We note that we must have $[z^{-1}]\left(\frac{d}{dz}( h_\ell(z) a(z))\right)=0$, so for $p-\ell-r \ge 0$, we see that $[z^{-1}]\left(\frac{d}{dz}( h_\ell(z) a(z))\right)=0$ is a linear combination of $[z^\ell]a(z),[z^{\ell+1}]a(z),\dots,[z^{\ell+r}]a(z)$ in which $[z^\ell]a(z)$ has nonzero coefficient $ (-1)^r  \prod_i y_i\left((\ell + r) - \sum_i m_i c\right)$. The latter coefficient is nonzero since $c$ is generic and none of the $y_i$ are $0$. Therefore $[z^\ell]a(z)$ is a linear combination of $[z^{\ell+1}]a(z),\dots,[z^{\ell+r}]a(z)$ when $\ell \le p-r$. 

Since $[z^{p-r+1}]a(z)=\dots=[z^p]a(z)=0$, by induction starting with $\ell=p-r$ we find that $[z^\ell]a(z)=0$ for $\ell=p-r,p-r-1,\dots,0$. Therefore, $[z^0]a(z)=0$; a contradiction. We conclude that $y_i=0$ for some $i$.

Assume without loss of generality that $y_{r-1}=0$. If $r>1$, we can repeat the same argument again with $y_0,\dots,y_{r-2}$. 

We are able to do this because our calculation of $F'(z)$ and $V(z)$ so that $z^p \mid V(z)$ relies only on the fact that $\sum m_iy_i=0$. Since $y_{r-1}=0$, we see that $\sum_{i=0}^{r-2}m_iy_i=\sum_{i=0}^{r-1}m_iy_i=0$. Then since the $y_i$ are distinct, $\prod_{i=0}^{r-2} y_i \ne 0$, so both necessary assumptions are true. We will then find that one of the $y_i$ with $i < r-1$ is $0$, which gives a contradiction since the $y_i$ are distinct. Therefore we must have $r=1$ and $y_0=0$. 
\end{proof}

\section{The proof of the main result}

\begin{theorem}\label{thm:main}
For generic $c$, the $f_i$ are linearly independent and generate the maximal ideal $J$ of the polynomial representation for $\HH_{1, c}(\h)$, and the irreducible quotient $L = A/J$ is a complete intersection with Hilbert series 
\[
h_L(t) = \left(\frac{1-t^p}{1-t}\right)^{n-1}.
\]
\end{theorem}
\begin{proof}
It suffices to show that the $f_i$ generate the ideal $J$ and that $A/J$ has Hilbert series $\left(\frac{1-t^p}{1-t}\right)^{n-1}$. 

By Propositions~\ref{prop:linind} and \ref{prop:ci}, $A/I$ is a complete intersection with $n-1$ linearly independent generators $f_i$ in degree $p$, so then its Hilbert series is $h_{A/I}(t)=\left(\frac{1-t^p}{1-t}\right)^{n-1}$. By Proposition~\ref{prop:ann}, the generators of $I$ are annihilated by the Dunkl operators, so $I \subseteq J$.

By \cite[Proposition 3.4]{BC1}, the Hilbert series of $A/J$ is $\left(\frac{1-t^p}{1-t}\right)^{n-1}h(t^p)$ for some polynomial $h$ with nonnegative integer coefficients. Since $I \subseteq J$, we see that $h_{A/I}(t) \ge h_{A/J}(t)$ coefficient-wise; hence $h(t)=1$. Therefore $h_{A/I}(t)=h_{A/J}(t)$, so $I=J$, completing the proof.
\end{proof}

\bibliographystyle{alpha}
\bibliography{rca-bib}

\end{document}
