\documentclass{amsart}
\usepackage{amsmath}
\usepackage{amsfonts}
\usepackage{amssymb,amsthm}
\usepackage{color}
\usepackage[margin=1in]{geometry}
\usepackage{graphicx}
\usepackage{verbatim}
\usepackage{hyperref}
\hypersetup{
    colorlinks,
    citecolor=blue,
    filecolor=black,
    linkcolor=black,
    urlcolor=black
}

\usepackage{todonotes}

\numberwithin{equation}{section}
\theoremstyle{definition}
\newtheorem{theorem}{Theorem}[section]
\newtheorem{proposition}[theorem]{Proposition}
\newtheorem{lemma}[theorem]{Lemma}
\newtheorem{corollary}[theorem]{Corollary}
\newtheorem{conjecture}[theorem]{Conjecture}
\newtheorem{problem}[theorem]{Problem}
\newtheorem*{remark}{Remark}
\newtheorem{definition}[equation]{Definition}
\newenvironment{defn}[1][]{%
    \begin{definition}[#1]\pushQED{\qed}}{\popQED \end{definition}}

\newcommand{\xx}[1]{\textcolor{blue}{#1}}
\newcommand{\zb}{\overline{z}}
\newcommand{\cc}[1]{\overline{#1}}
\newcommand{\m}[1]{\left|#1\right|}
\newcommand{\Res}{\operatorname{Res}}
\newcommand{\p}[1]{\left(#1\right)}
\newcommand{\rank}{\operatorname{rank}}
\newcommand{\sgn}{\operatorname{sgn}}
\newcommand{\D}{\triangle}
\newcommand{\ZZ}{\mathbb{Z}}
\newcommand{\f}[1]{\sum_{n=-\infty}^\infty #1_ne^{inx}}
\newcommand{\fn}[1]{\sum_{n=-\infty}^\infty #1e^{inx}}
\newcommand{\BB}{\mathcal{B}}
\newcommand{\M}{\mathcal{M}}
\newcommand{\mm}[1]{\m{\m{#1}}}
\newcommand{\ang}[1]{\left\langle #1 \right\rangle}
\newcommand{\h}{\mathfrak{h}}
\newcommand{\GL}{{\rm GL}}
\newcommand{\HH}{\mathcal{H}}
\newcommand{\arxiv}[1]{\href{http://arxiv.org/abs/#1}{{\tt arXiv:#1}}}
\newcommand{\Sym}{\operatorname{Sym}}
\renewcommand{\char}{\text{char}}
\newcommand{\sspan}{\text{span}}

\begin{document}
\title[Polynomial representation of type $A_{n - 1}$ rational Cherednik algebra in characteristic $p \mid n$]{The polynomial representation of the type $A_{n - 1}$ rational Cherednik algebra in characteristic $p \mid n$}
\author{Sheela Devadas}
\email[S. Devadas]{sheelad@mit.edu}
\author{Yi Sun}
\email[Y. Sun]{yisun@math.mit.edu}
\date{\today}

\begin{abstract}
We study the polynomial representation of the rational Cherednik algebra of type $A_{n-1}$ with generic parameter in characteristic $p$ for $p \mid n$. We give explicit formulas for generators for the maximal ideal, show that they cut out a complete intersection, and thus compute the Hilbert series of the irreducible quotient. Our methods are motivated by taking characteristic $p$ analogues of existing characteristic $0$ results.
\end{abstract}

\maketitle
\setcounter{tocdepth}{1}
\tableofcontents

\section{Introduction}

The present work presents a detailed study of the polynomial representation of the type $A_{n - 1}$ rational Cherednik algebra over a field of characteristic $p$ dividing $n$.  Rational Cherednik algebras were introduced by Etingof-Ginzburg in \cite{EG} as a rational degeneration of the double affine Hecke algebra dependent on two parameters $\hbar$ and $c$.  In characteristic $0$, their type $A$ representation theory has been the subject of extensive study.  We refer the reader to \cite{EM} for a survey of these results. 

In characteristic $p$ and especially in the modular case, much less is known about the representation theory of the rational Cherednik algebra.  In this paper, we consider the modular case $p \mid n$.  For $\hbar = 1$ and generic $c$, we provide a complete characterization of the irreducible quotient of the polynomial representation.  We give explicit generators for the unique maximal ideal, show that the irreducible quotient is a complete intersection, and compute its Hilbert series.

Our techniques are inspired by taking characteristic $p$ analogues of results about Cherednik algebras in characteristic $0$.  In particular, our explicit expression for generators of the maximal ideal was obtained by converting expressions containing complex residues to equivalent expressions dealing only with formal power series which may be interpreted in characteristic $p$.  While we restrict our present study to the polynomial representation in type $A$, we view this problem as a test case for this philosophy and believe it may admit wider application.

In the remainder of this introduction, we give precise statements of our results and explain how they relate to other recent work.

\subsection{The rational Cherednik algebra in positive characteristic}

Throughout this paper, we work over an algebraically closed field $k$ of characteristic $p > 0$ and fix $n$ so that $p \mid n$.  Let $S_n$ denote the symmetric group on $n$ elements, $V = k^n$ its permutation representation, and $s_{ij} \in S_n$ the transposition permuting $i$ and $j$.  Fix a basis $y_1,\ldots,y_{n}$ for $V$ and a dual basis $x_1, \ldots, x_{n}$ for $V^*$.  Let $\h$ and $\h^*$ be the dual $(n - 1)$-dimensional $S_n$-representations which are subrepresentation and quotient of $V$ and $V^*$, respectively given by
\[
\h = \sspan\{y_i - y_j \mid i \neq j\} \text{ and } \h^* = V^*/(x_1 + \cdots + x_{n}).
\]
The action of $S_n$ on $\h$ and $\h^*$ is given explicitly by natural permutation of basis vectors.

Fix constants $\hbar$ and $c$ in $k$.  Denoting the tensor algebra of $\h \oplus \h^*$ by $T(\h \oplus \h^*)$, the \textit{type $A_{n - 1}$ rational Cherednik algebra} $\HH_{\hbar, c}(\h)$ is the quotient of $k[S_n] \ltimes T(\h \oplus \h^*)$ by the relations
\begin{align*}
[x_i,x_j]=0, \quad [y_i,y_j] = 0, \quad [y_i,x_j] = \hbar\langle y_i,x_j\rangle - \sum_{m=1}^{n}\sum_{l=1}^{n} c\langle y_i,x_m-x_l \rangle\langle y_m-y_l,x_j\rangle s_{ml}
\end{align*}
for all $1 \le i,j \le n$.  There is a $\ZZ$-grading on $\HH_{\hbar,c}(\h)$ given by setting $\deg x=1$ for $x \in \h^*$, $\deg y = -1$ for $y \in \h$, and $\deg g=0$ for $g \in k[S_n]$.  In addition, $\HH_{\hbar, c}(\h)$ admits a PBW decomposition 
\[
\HH_{\hbar,c}(\h) = \Sym(\h) \otimes_k k[S_n] \otimes_k \Sym(\h^*).
\]
For any $\alpha \ne 0$, $\HH_{\hbar,c}(\h)$ and $\HH_{\alpha\hbar,\alpha c}(\h)$ are isomorphic as algebras, so only the cases $\hbar = 0$ or $\hbar = 1$ need be considered.  In this paper, we restrict our attention to $\hbar = 1$.

\subsection{Polynomial representation of the rational Cherednik algebra}

The rational Cherednik algebra $\HH_{\hbar, c}(\h)$ admits a $\ZZ_{\geq 0}$-graded representation on the polynomial ring $A = \Sym(\h^*)$, known as the \textit{polynomial representation}.  The actions of $\Sym(\h^*)$ and $k[S_n]$ on $A$ are by left multiplication and the $S_n$ action on $\h^*$, respectively.  The action of $\Sym(\h)$ is implemented by letting $y \in \h$ act via the \textit{Dunkl operator} $D_y$ defined by
\begin{align*}
D_y(f) = \hbar \partial_y f  - \sum_{m=1}^{n} \sum_{l=1}^{n} c  \langle y, x_m-x_l \rangle \frac{(1-s_{ml}) f}{x_m - x_l},
\end{align*}
where we note that $\frac{(1-s_{ml}) f}{x_m - x_l}$ is a polynomial for $f \in A$. Explicitly, we have  
\[
D_{y_i - y_j} = \hbar \partial_{y_i-y_j}-c\sum_{m \ne i} \frac{1-s_{mi}}{x_i-x_m}+c\sum_{m \ne j} \frac{1-s_{mi}}{x_j-x_m}
\]

where $\partial_{y_i-y_j}$ is a differential operator we compute by letting $\partial_{y_i-y_j}(x) = \langle y_i-y_j,x\rangle$ for all $x \in \h^*$ and extending via the Leibnitz rule to the rest of $A=\Sym(\h^*)$. 

We note that we can extend the Dunkl operators $D_y$ to formal power series in $z$ with coefficients from $A$ by letting the Dunkl operators act on each coefficient of the power series. 

\subsection{Maximal ideal and irreducible quotient of the polynomial representation}

As described in \cite[Section 2.5]{BC1}, there is a contravariant form 
\[
\beta_c: \Sym(\h^*) \otimes \Sym(\h) \to k
\]
defined by setting $\beta_c(1, 1) = 1$ and imposing for all $x \in \h^*, y \in \h, f \in \Sym(\h^*), g \in \Sym(\h)$ that
\[
\beta_c(xf,g)=\beta_c(f,xg) \qquad \text{ and } \qquad \beta_c(f,yg) = \beta_c(yf,g).
\]
Then the polynomial representation $\Sym(\h^*)$ has unique maximal graded proper submodule $J = \ker(\beta_c)$.  By the definition of $\beta_c$, $J$ is generated by homogeneous vectors $f \in A$ killed by all Dunkl operators $D_y$.  Such $f$ are known as \textit{singular vectors}. 

The quotient $L = A/J$ is an irreducible representation of $\HH_{\hbar,c}(\h)$.  It inherits a $\ZZ_{\geq 0}$-grading from $A$, and we recall that its Hilbert series is 
\[
h_L(t) = \sum_{j \geq 0} \dim L_j t^j,
\]
where $L_j$ is the degree $j$ subspace of $L$. 

\subsection{Statement of the main result}

For a formal Laurent series $r(z)$, we denote by $[z^l] r(z)$ the coefficient of $z^l$ in $r(z)$. We present some power series as rational functions which we expand near $z=0$. For $i = 1, \ldots, n - 1$, define the formal power series
\[
F_i(z)=\frac{1}{1-x_iz} \left(\sum_{m=0}^{p-1} \binom{c}{m}(-1+\prod_{j=1}^{n} (1-x_jz))^m\right),
\]
where $\binom{c}{m} = \frac{c (c - 1) \cdots (c - m + 1)}{m!}$ and consider the coefficients $f_i = [z^p] F_i(z)$.

\newtheorem*{thm:main}{Theorem \ref{thm:main}} \begin{thm:main}
For generic $c$, the $f_i$ are linearly independent and generate the maximal ideal $J$ of the polynomial representation for $\HH_{1, c}(\h)$, and the irreducible quotient $L = A/J$ is a complete intersection with Hilbert series 
\[
h_L(t) = \left(\frac{1-t^p}{1-t}\right)^{n-1}.
\]
\end{thm:main}
\begin{remark}
In Theorem \ref{thm:main}, by generic $c$ we mean $c$ avoiding countably many values.
\end{remark}

\subsection{Connections to previous work}

Our study is motivated by previous work on the representation theory of the type $A$ rational Cherednik algebra in both characteristic $0$ and $p$.  The type $A$ non-modular case $p \gg n$ was studied in \cite{BFG}, and some properties of the maximal ideal of the polynomial representation were given in both modular and non-modular cases in \cite{BC1}.  In the modular case $p \mid n$, for $p = 2$, the polynomial representation associated to the $n$-dimensional permutation representation was studied.
\begin{theorem}[{\cite[Theorem 5.1]{L}}] \label{thm:lian}
The irreducible quotient of the polynomial representation associated to the $n$-dimensional reflection representation is a complete intersection with Hilbert series
\[
h(t) = (1 + t)^n (1 + t^2).
\]
The corresponding maximal ideal is generated by $n - 1$ elements of degree $2$ and one element of degree $4$. 
\end{theorem} 
It was further conjectured by Lian in \cite[Conjecture 5.2]{L} that for all $p$ the corresponding irreducible is a complete intersection with maximal ideal having $n - 1$ generators in degree $p$ and a single generator in degree $p^2$.  Our results are consistent with the restriction of Lian's conjecture to the case when $\h$ is restricted to the $(n - 1)$-dimensional quotient.  It would be interesting to extend our work to prove Lian's conjecture in full.  For general $p \mid n$, a subideal of the maximal ideal was computed in \cite[Proposition 6.1]{DS}.

In characteristic $0$, our results parallel the explicit decomposition of the polynomial representation of the type $A$ rational Cherednik algebra given in \cite{BEG, CE}.  There, the polynomial representation is irreducible unless $c = \frac{r}{n}$ for some integer $r$, and an explicit set of generators of the maximal ideal is known.
\begin{proposition}[{\cite[Proposition 3.1]{CE}}] \label{prop:ce}
If $\char(k) = 0$ and $c = \frac{r}{n}$, the maximal ideal $J \subset A$ of the polynomial representation $A$ of $\HH_{\hbar,c}(\h)$ is generated by
\[
\Res_\infty\left[\frac{dz}{z-x_j} \prod_{i=1}^{n} (z-x_i)^c\right] \text{ for $j=1,\dots,n-1$}.
\]
\end{proposition}
We interpret the characteristic $p$ analogue of Proposition \ref{prop:ce} to mean that if $r = p$ and $p \mid n$, then taking $c = \frac{p}{n}$ should correspond to taking generic $c$.  While this substitution is of course invalid, Proposition \ref{prop:ce} may be interpreted as a statement about certain formal power series.  By using a power series version of this construction of generators which makes sense in characteristic $p$, we are able to mimic the arguments of \cite{BEG, CE} to show that they cut out a complete intersection and generate the entire ideal.  We believe that the philosophy of taking characteristic $p$ analogues of characteristic $0$ results for the rational Cherednik algebra should apply more generally and hope to explore this further in future work.

\subsection{Outline of the paper}

The remainder of this paper is organized as follows.  In Section 2, we check that the generators $f_i$ are linearly independent singular vectors.  In Section 3, we show that they cut out a complete intersection.  In Section 4, we put these facts together to conclude Theorem \ref{thm:main}.

\subsection{Acknowledgements} 

The authors thank P. Etingof for suggesting the problem and for helpful discussions.  Some exploratory computations were done using Sage.  S.~D. was supported by the MIT Undergraduate Research Opportunities Program (UROP). Y.~S. was supported by a NSF Graduate Research Fellowship (NSF Grant \#1122374).

\todo[inline]{Ask Pavel for his grant number}

\section{An explicit construction of singular vectors}

\subsection{Definition of the singular vectors}

Define the polynomials 
\[
g(z)=\prod_{j=1}^{n} (1-x_jz) \qquad \text{ and } \qquad F(z) = \sum_{m=0}^{p-1} \binom{c}{m} (g(z)-1)^m.
\]
In terms of $g(z)$ and $F(z)$, we have $F_i(z) = \frac{F(z)}{1-x_iz}$ and $f_i = [z^p] \frac{F(z)}{1-x_iz}$.  We will show that $f_i$ are singular vectors.

\subsection{Computation of some partial derivatives}

We begin by computing some partial derivatives of $F$ which will be useful for computing the action of the Dunkl operators.

\begin{lemma}\label{lem:z2g}
We have $[z^0]g(z)=1$ and $[z^1]g(z)=0$, meaning $z^2 \mid g(z)-1$.
\end{lemma}
\begin{proof}
Consider the expansion 
\[
g(z)=\prod_{j=1}^{n} (1-x_jz) = 1-z\sum_i x_i+z^2e_2(x_1, \ldots, x_{n})+\dots+(-1)^nz^ne_n(x_1, \ldots, x_{n})
\]
for the elementary symmetric polynomials $e_2, \ldots, e_n$.  Recalling that $\sum_i x_i=0$ by hypothesis, we see that $[z^1]g(z)=0$ and $[z^0]g(z)=1$, so $z^2 \mid g(z)-1$ as desired.
\end{proof}

\begin{lemma}\label{lem:dFdz}
We have
\[
F'(z) = V(z) - \sum_{j=1}^{n} \frac{cx_j}{1-x_jz} F(z)
\]
such that $V(z)$ is a power series with $[z^l]V(z)=0$ for $l=0,\dots,p-1$. 
\end{lemma}

\begin{proof}
We see easily that 
\[
\frac{\partial g}{\partial z} = g(z) \sum_j \frac{-x_j}{1-x_jz}.
\]
We now consider $\frac{\partial F}{\partial z}$. We compute
\begin{align*}
\frac{\partial F}{\partial z}
&=\sum_{m=1}^{p-1}m\binom{c}{m}(g(z)-1)^{m-1}\frac{\partial g}{\partial z}\\
&=-\sum_j \frac{x_j}{1-x_jz}\sum_{m=0}^{p-2}c\binom{c-1}{m}(g(z)-1)^{m}(g(z)-1+1)\\
&=-\sum_j \frac{x_j}{1-x_jz}\left(\sum_{m=0}^{p-2}c\binom{c-1}{m}(g(z)-1)^{m}+\sum_{m=1}^{p-1}c\binom{c-1}{m-1}(g(z)-1)^{m}\right)\\
&=-\sum_j \frac{x_j}{1-x_jz}\left(-c\binom{c-1}{p-1}(g(z)-1)^{p-1}+\sum_{m=0}^{p-1}c\binom{c}{m}(g(z)-1)^{m}\right)\\
&=-\sum_j \frac{cx_j}{1-x_jz}F(z)+\sum_j \frac{x_j}{1-x_jz}c\binom{c-1}{p-1}(g(z)-1)^{p-1}.
\end{align*}
Defining the formal power series
\[
V(z)=\sum_j \frac{x_j}{1-x_jz}c\binom{c-1}{p-1}(g(z)-1)^{p-1},
\]
we see that 
\[
F'(z) = V(z) - \sum_{j=1}^{n} \frac{cx_j}{1-x_jz} F(z).
\]
It remains only to show that $[z^l]V(z)=0$ for $l=0,\dots,p-1$.  By definition of $V(z)$, we see that $(g(z)-1)^{p-1} \mid V(z)$, hence by Lemma~\ref{lem:z2g}, $z^{2p-2} \mid V(z)$. Since $p \ge 2$, this implies $z^p \mid V(z)$, completing the proof.
\end{proof}

\begin{lemma}\label{lem:dFdxi} We have
\[
\partial_{y_2-y_1}(F(z))= G(z) - \left(\frac{zc}{1-x_2z}-\frac{zc}{1-x_1z}\right)F(z)
\]
such that $G(z)$ is a power series with $[z^l]G(z)=0$ for $l=0,\dots,p$. 
\end{lemma}

\begin{proof}

We see that
\[
\partial_{y_2-y_1}(g(z))=g(z)\left(-\frac{z}{1-x_2z}+\frac{z}{1-x_1z}\right).
\]
Then we see that
\begin{align*}
\partial_{y_2-y_1}(F(z))&=\left(\sum_{m=1}^{p-1}m\binom{c}{m}(g(z)-1)^{m-1}\right)\partial_{y_2-y_1}(g(z))\\
&=\left(-\frac{z}{1-x_2z}+\frac{z}{1-x_1z}\right)\left(\sum_{m=1}^{p-1}m\binom{c}{m}(g(z)-1)^{m-1}\right)g(z)\\
&=\left(-\frac{z}{1-x_2z}+\frac{z}{1-x_1z}\right)\left(\sum_{m=0}^{p-2}c\binom{c-1}{m}(g(z)-1)^{m}+\sum_{m=0}^{p-2}c\binom{c-1}{m}(g(z)-1)^{m+1}\right)\\
&=\left(-\frac{zc}{1-x_2z}+\frac{zc}{1-x_1z}\right)\left(-\binom{c-1}{p-1}(g(z)-1)^{p-1}+F(z)\right).
\end{align*}
We define
\[
G(z)=\left(-\frac{zc}{1-x_2z}+\frac{zc}{1-x_1z}\right)\left(-\binom{c-1}{p-1}(g(z)-1)^{p-1}\right).
\]
Then we see that
\[
\partial_{y_2-y_1}(F(z))= G(z) - \left(\frac{zc}{1-x_2z}-\frac{zc}{1-x_1z}\right)F(z)
\]
It remains only to show that $[z^l]G(z)=0$ for $l=0,\dots,p$. By definition of $G(z)$, we see that $z(g(z)-1)^{p-1} \mid G(z)$, so by Lemma~\ref{lem:z2g}, $z^{2p-1} \mid G(z)$. Since $p \ge 2$, this implies $z^{p+1} \mid G(z)$, completing the proof.
\end{proof}

\subsection{Proving $f_i$ are singular vectors} 

\begin{proposition}\label{prop:ann} 
The elements $f_i$ for $i=1,\dots,n-1$ are singular vectors in $A$.
\end{proposition}
\begin{proof}
We must show that $f_i$ is annihilated by annihilated by $D_j-D_l$ for all $j \ne l$.

We see that for all $i$, $f_i$ is symmetric in the $x_j$ for $j \ne i$. It therefore suffices to consider the action of the Dunkl operators on $f_1$. We note that the Dunkl operators $D_{y_i-y_j}$ for all $i \ne j$ are spanned by $D_{y_i-y_1}$ for $1<i \le n$; then because $f_1$ is symmetric in the $x_i$ with $i > 1$, we see that it suffices to consider the action of $D_{y_2-y_1}$ on $f_1$ to show that the $f_i$ are annihilated by $D_{y_i-y_j}$ for all $i \ne j$. 

By Lemma~\ref{lem:dFdxi} we recall that $\partial_{y_2-y_1}(F(z))= G(z) - \left(\frac{zc}{1-x_2z}-\frac{zc}{1-x_1z}\right)F(z)$ for a power series $G(z)$ with $[z^l]G(z)=0$ for $l=0,\dots,p$. 

We can now calculate $\partial_{y_2-y_1}(F_1(z))$. We compute
\begin{align*}
\partial_{y_2-y_1}(F_1(z))
&=-\frac{z}{(1-x_1z)^2}F(z)+\frac{1}{1-x_1z}\partial_{y_2-y_1}(F(z))\\
&=-\frac{z}{1-x_1z}F_1(z)+\frac{1}{1-x_1z}\left(\frac{zc}{1-x_1z}-\frac{zc}{1-x_2z}\right)F(z)+\frac{G(z)}{1-x_1z}\\
&=\left(\frac{z(c-1)}{1-x_1z}-\frac{zc}{1-x_2z}\right)F_1(z)+\frac{G(z)}{1-x_1z}.
\end{align*}
We note that $F_1(z)$ is invariant under $s_{ij}$ where $1 < i,j$. Therefore for transpositions we need only consider transpositions of the form $s_{1i}$ for $1 < i \le n$. We see that 
\begin{align*}
\frac{1-s_{1i}}{x_1-x_i}(F_1(z))
&=\frac{1}{x_1-x_i}\left(\frac{1}{1-x_1z}-\frac{1}{1-x_iz}\right)F(z)\\
&=\frac{z}{(1-x_iz)(1-x_1z)}F(z)\\
&=\frac{z}{1-x_iz}F_1(z).
\end{align*}
We recall that we need only consider the action of $D_{y_2-y_1}$ on $F_1(z)$. We see that 
\[
D_{y_2-y_1}=\partial_{y_2-y_1}-c\frac{1-s_{12}}{x_2-x_1}+c\sum_{j > 1} \frac{1-s_{1j}}{x_1-x_j}
\]
since $F_1(z)$ is invariant under $s_{ij}$ where $1 < i,j$. We now compute
\begin{align*}
D_{y_2-y_1}(F_1(z))&=\left(\partial_{y_2-y_1}-c\frac{1-s_{12}}{x_2-x_1}+c\sum_{j > 1} \frac{1-s_{1j}}{x_1-x_j}\right)(F_1(z))\\
&=\partial_{y_2-y_1}(F_1(z))-c\frac{1-s_{12}}{x_2-x_1}(F_1(z))+c\sum_{j > 1} \frac{1-s_{1j}}{x_1-x_j}(F_1(z))\\
&=\frac{G(z)}{1-x_1z}+\left(\frac{z(c-1)}{1-x_1z}-\frac{zc}{1-x_2z}\right)F_1(z)+\frac{zc}{1-x_2z}F_1(z)+\sum_{j > 1} \frac{zc}{1-x_jz}F_1(z)\\
&=\frac{G(z)}{1-x_1z}+\frac{z(c-1)}{1-x_1z}F_1(z)+\sum_{j>1} \frac{zc}{1-x_jz}F_1(z).
\end{align*}

In order to show that the $p$th coefficient in this formal power series is $0$, we must consider $\frac{\partial F_1}{\partial z}$. 

We recall by Lemma~\ref{lem:dFdz} that
\[
\frac{\partial F}{\partial z}=V(z)+\left(\sum_j \frac{-cx_j}{1-x_jz}\right)F(z).
\]
where $[z^l]V(z)=0$ for $l=0,\dots,p-1$. 

From this it follows that
\begin{align*}
\frac{\partial F_1}{\partial z} &= \frac{\partial}{\partial z}\left(\frac{F(z)}{1-x_1z}\right)\\
&=\frac{1}{1-x_1z} \frac{\partial F}{\partial z}+\frac{x_1}{(1-x_1z)^2} F(z)\\
&=\frac{V(z)}{1-x_1z}+\frac{1}{1-x_1z}\left(\sum_j \frac{-cx_j}{1-x_jz}\right)F(z)+\frac{x_1}{(1-x_1z)^2} F(z)\\
&=\frac{V(z)}{1-x_1z}+\left(\sum_j \frac{-cx_j}{1-x_jz}\right)F_1(z)+\frac{x_1}{1-x_1z} F_1(z).
\end{align*}
We now again consider $D_{y_2-y_1}(F_1(z))$. Recall that $n \equiv 0 \bmod p$, so in particular we can add $n$ times any multiple of $F_1(z)$ since that is $0$ in characteristic $p$. 
\begin{align*}
D_{y_2-y_1}(F_1(z))&=\frac{G(z)}{1-x_1z}+\frac{z(c-1)}{1-x_1z}F_1(z)+\sum_{j>1} \frac{zc}{1-x_jz}F_1(z)\\
&=\frac{G(z)}{1-x_1z}-zF_1(z)+zF_1(z)-\frac{z}{1-x_1z}F_1(z)+\left(\sum_{j} -zcF_1(z)+\frac{zc}{1-x_jz}F_1(z)\right)\\
&=\frac{G(z)}{1-x_1z}-zF_1(z)+\frac{-x_1z^2}{1-x_1z}F_1(z)+\left(\sum_{j} \frac{x_jcz^2}{1-x_jz}F_1(z)\right)\\&=\frac{G(z)}{1-x_1z}-zF_1(z)-z^2\frac{\partial F_1}{\partial z}+z^2\frac{V(z)}{1-x_1z}\\
&=\frac{G(z)+z^2V(z)}{1-x_1z}-zF_1(z)-z^2\frac{\partial F_1}{\partial z}.
\end{align*}

We note that $[z^p]\left(\frac{G(z)+z^2V(z)}{1-x_1z}\right)$ is a linear combination of $[z^l](G(z)+z^2V(z))$ for $0 \le l \le p$. This means that $[z^p]\left(\frac{G(z)+z^2V(z)}{1-x_1z}\right)$ is a linear combination of $[z^l]G(z)$ for $0 \le l \le p$ and $[z^l]V(z)$ for $0 \le l \le p-2$. By Lemmas~\ref{lem:dFdz} and \ref{lem:dFdxi} we recall that the coefficients of $G(z),V(z)$ under consideration are all $0$. This implies that $[z^p]\left(\frac{G(z)+z^2V(z)}{1-x_1z}\right)=0$. Therefore
\[
[z^p](D_{y_2-y_1}(F_1(z)))=[z^p]\left(-zF_1(z)-z^2\frac{\partial F_1}{\partial z}\right).
\]
Let $b=[z^{p-1}](F_1(z))$. We see that $[z^p](-zF_1(z))=-b$. Then $[z^{p-2}]\left(\frac{\partial F_1}{\partial z}\right)=(p-1)b=-b$, so $[z^p]\left(-z^2\frac{\partial F_1}{\partial z}\right)=b$. Therefore $[z^p]\left(-zF_1(z)-z^2\frac{\partial F_1}{\partial z}\right)=-b+b=0$, so 
\[
[z^p](D_{y_2-y_1}(F_1(z)))=D_{y_2-y_1}([z^p]F_1(z))=D_{y_2-y_1}f_1=0.
\]
By the symmetry of the $f_i$, this is the only Dunkl operator we need consider; it is clear that all the $f_i$ are killed by the Dunkl operators.
\end{proof}

\subsection{Proof of linear independence of $f_i$}

\begin{proposition}\label{prop:linind} 
For generic $c$, the $f_i$ for $i=1,\dots,n-1$ are linearly independent degree $p$ homogeneous polynomials.
\end{proposition} 
\begin{proof} 
We have the expansion
\[
F_i(z)=\frac{1}{1-x_iz}\sum_{m=0}^{p-1} \binom{c}{m} (g(z)-1)^m=\sum_{m=0}^\infty x_i^mz^m\sum_{m=0}^{p-1} \binom{c}{m} (g(z)-1)^m.
\]
Because for any $l$ the coefficient of $z^l$ in each factor is a homogeneous polynomial of degree $l$, we see that $[z^p]F_i(z)$ is homogeneous of degree $p$.

For linear independence, suppose that $\sum_{i=1}^{n-1} \lambda_if_i=0$ for some $\lambda_i \in k$.  Substitute $x_{n} = -1$, $x_j=1$ and $x_i=0$ for $i \ne j,i < n$ so that $g(z)=(1-z)(1+z)=1-z^2$ and hence
\[
F_j(z)=\sum_{m=0}^\infty z^m\sum_{m=0}^{p-1} \binom{c}{m} (-z^2)^m
\]
and 
\[
F_i(z)=\sum_{m=0}^\infty 0^mz^m\sum_{m=0}^{p-1} \binom{c}{m} (-z^2)^m=\sum_{m=0}^{p-1} \binom{c}{m} (-z^2)^m \text{ for $i < n - 1$, $i \neq j$}.
\]
If $p = 2$, we see that $[z^2]F_j(z)=1-c$ and $[z^2]F_i(z)=-c$, so varying $j$ implies that
\[
\lambda_j = c\sum_{i=1}^{n-1} \lambda_i \text{ for all }j.
\]
In particular, all $\lambda_i$ have common value $\lambda \in k$ solving $(1-c(n-1))\lambda = 0$, which for generic $c$ implies that $\lambda = 0$, giving linear independence.

If $p > 2$, we have 
\[
[z^p]F_j(z)=f_j=\sum_{m=0}^{(p-1)/2} (-1)^m\binom{c}{m}=\binom{c-1}{(p-1)/2}
\]
and $[z^p]F_i(z)=f_i=0$ for $i < n$ and $i \neq j$. For generic $c$, we have $\binom{c-1}{(p-1)/2} \ne 0$, meaning that 
\[
\sum_{i=1}^{n-1} \lambda_if_i=0 = \lambda_j\binom{c-1}{(p-1)/2}=0
\]
implies $\lambda_j=0$.  Varying $j$ implies that $\lambda_j=0$ for all $j$, again yielding linear independence.
\end{proof}

\section{Complete intersection properties}

Consider the ideal
\[
I = \langle f_1, \ldots, f_{n - 1} \rangle \subset A
\]
generated by the $f_i$.  In this section, we will show that $A/I$ is a complete intersection, for which we require some preparatory computations.

\subsection{Computations of derivatives}

Our goal in this subsection is to establish Lemma \ref{lem:resbyparts}, which is a formal power series analogue of \cite[Lemma 3.4]{CE}.  Our method will be an adaptation of the ``residues by parts'' argument which appears there.  First, we require some direct computations of derivatives.

For any $s_1,\dots,s_{r}$ distinct elements of $k$ and positive integers $m_1,\dots,m_r$, we define 
\[
g(z)=\prod_{i=1}^{r} (1-s_iz)^{m_i} \text{  and } F(z)=\sum_{m=0}^{p-1} \binom{c}{m}(g(z)-1)^m
\]
as an analogue of $g(z)$ and $F(z)$ from the previous section. We also let $a(z)$ be the formal power series
\[
a(z)=\frac{1}{\prod_{i=1}^{r}(1-s_iz)}F(z)
\]
and define the polynomials $b_i(z) = \prod_{j \neq i} (1 - s_j z)$ for $i=1, \dots, r$. 

\begin{lemma}\label{lem:vand}Let $s_1, \ldots, s_{r }$ be distinct elements of $k$.  Then the polynomials $b_i(z) = \prod_{j \neq i} (1 - s_j z)$ are linearly independent.
\end{lemma}
\begin{proof} 
Suppose that for some $\lambda_i \in k$ we have $\sum_i \lambda_ib_i(z)=0$.  Then $\left(\prod_j (1-s_jz)\right)\left(\sum_i \frac{\lambda_i}{1-s_iz}\right)$ must be $0$ in the ring of formal power series. Since the ring of power series is an integral domain, $\sum_i \frac{\lambda_i}{1-s_iz}=0$. In particular for $l=0,\dots,r-1$ we have $\sum_i \lambda_is_i^l=0$.  However, the vectors $(1,s_i,\dots,s_i^{r-1})$ are linearly independent by the Vandermonde determinant, so $\lambda_i=0$ for all $i$, meaning that $b_i(z)$ are linearly independent.
\end{proof}

\begin{corollary}\label{corr:dFdz2}
Let $s_1,\dots,s_{r}$ be distinct elements of $k$ and $m_i$ positive integers such that $\sum_i m_is_i=0$.  
For some $V(z)$ satisfying $[z^l]V(z)=0$ for $0 \le l \le p-1$, we have
\[
F'(z)=V(z) - \sum_j \frac{cm_js_j}{1-s_jz}F(z).
\]
\end{corollary}
\begin{proof}
Evaluate Lemma \ref{lem:dFdz} by substituting $m_i$ copies of $s_i$ for the $x_i$.
\end{proof}

\begin{lemma} \label{lem:resbyparts}
Let $s_1,\dots,s_{r}$ be distinct non-zero elements of $k$ and $m_i$ positive integers so that that $\sum_i m_is_i=0$.  For any integer $l \le p-r$, $[z^l]a(z)$ is a linear combination of $[z^{l+1}]a(z)$, $\ldots$, $[z^{l+r}]a(z)$.
\end{lemma}
\begin{proof} 
We will use an analogue of ``residues by parts'' in a formal power series setting.   We begin by showing that 
\[
a'(z)=z^pb(z)-\sum_j \frac{(m_jc-1)s_j}{1-s_jz}a(z)
\]
for some formal power series $b(z)$. By Corollary~\ref{corr:dFdz2} we have
\[
F'(z)=V(z)+\left(\sum_j \frac{-cm_js_j}{1-s_jz}F(z)\right),
\]
where $[z^l]V(z)=0$ for $0 \le l \le p-1$. 
We now compute
\begin{align*}
a'(z)&=\frac{d}{dz}\left(\frac{1}{\prod_{i=1}^{r}(1-s_iz)}F(z)\right)\\
&=\frac{1}{\prod_{i=1}^{r}(1-s_iz)}F'(z)+\sum_{j=1}^{r}\frac{s_j}{(1-s_jz)\prod_{i=1}^{r}(1-s_iz)}F(z)\\
&=\frac{1}{\prod_{i=1}^{r}(1-s_iz)}\left(V(z)+\sum_j \frac{-cm_js_j}{1-s_jz}F(z)\right)+\sum_{j=1}^{r}\frac{s_j}{1-s_jz}a(z)\\
&=\frac{V(z)}{\prod_{i=1}^{r}(1-s_iz)}+\sum_j \frac{-cm_js_j}{1-s_jz}a(z)+\sum_{j=1}^{r}\frac{s_j}{1-s_jz}a(z)\\
&=\frac{V(z)}{\prod_{i=1}^{r}(1-s_iz)}-\sum_j \frac{(m_jc-1)s_j}{1-s_jz}a(z).
\end{align*}
Since $[z^l]V(z)=0$ for $0 \le l \le p-1$ we see that 
\[
\frac{V(z)}{\prod_{i=1}^{r}(1-s_iz)}=z^pb(z)
\]
for some power series $b(z)$, giving the claim.

Now, for any integer $l$, define the Laurent polynomial 
\[
h_l(z)=z^{-l-r}\prod_i(1-s_iz).
\]
Notice that
\begin{align*}
\frac{d}{dz}( h_l(z) a(z)) &= -(l + r)z^{-1} h_l(z) a(z) - \sum_i \frac{s_i}{1 - s_iz} h_l(z) a(z) + h_l(z) a'(z) \\
&=  -(l + r)z^{-1} h_l(z) a(z) - \sum_i \frac{s_i}{1 - s_iz} h_l(z) a(z) - h_l(z) \sum_j \frac{(m_j c - 1)s_i}{1 - s_ijz} a(z)  + z^ph_l(z) b(z)\\
&= -(l + r)z^{-1} h_l(z) a(z) - \sum_i \frac{s_im_i c}{1 - s_iz} h_l(z) a(z) + z^ph_l(z) b(z)\\
&= -\left((l + r) z^{-1} + \sum_i \frac{s_im_i c}{1 - s_iz}\right) h_l(z) a(z) + z^ph_l(z) b(z).
\end{align*}
We note that 
\[
-\left((l + r) z^{-1} + \sum_i \frac{s_im_i c}{1 - s_iz}\right) h_l(z)
\]
is a Laurent polynomial with lowest degree term 
\[
- z^{-l-r-1} \left(l + r + \sum_i s_i m_i c\right) = - z^{-l-r-1}(l+ r)
\]
and highest degree term 
\[
- z^{-l - 1} (-1)^r  \prod_i s_i\left((l + r) - \sum_i m_i c\right).
\]
The lowest degree term of the formal power series $z^ph_l(z)b(z)$ has degree at least $p-l-r$. We always have 
\[
[z^{-1}]\left(\frac{d}{dz}( h_l(z) a(z))\right)=0,
\]
so for $p-l-r \ge 0$, we see that $[z^{-1}]\left(\frac{d}{dz}( h_l(z) a(z))\right)=0$ is a linear combination of $[z^l]a(z),[z^{l+1}]a(z),\dots,[z^{l+r}]a(z)$ in which $[z^l]a(z)$ has nonzero coefficient 
\[
(-1)^r  \prod_i s_i\left((l + r) - \sum_i m_i c\right),
\]
which is nonzero for $c$ generic and $\prod_i s_i \ne 0$. Therefore $[z^l]a(z)$ is a linear combination of $[z^{l+1}]a(z)$, $\dots$, $[z^{l+r}]a(z)$ when $l \le p-r$. 
\end{proof}

\subsection{Complete intersection property}

We now show that $A/I$ is a complete intersection.  Our technique is a translation of the idea of the proof of \cite[Theorem 3.2]{CE} to the formal power series context.

\begin{proposition}\label{prop:ci}
For generic $c$, the quotient $A/I$ is a complete intersection. 
\end{proposition}
\begin{proof}
It suffices to show that if $x_1,\dots,x_{n} \in k$ satisfy $f_i(x_1,\dots,x_{n})=0$ for all $i$, then $x_1=\dots=x_{n}=0$.  Suppose that the $x_i$ take the distinct values $\{s_1,\dots,s_{r}\}$, where $s_i$ occurs with multiplicity $m_i>0$ so that $g(z)=\prod_i (1-s_iz)^{m_i}$.  In these terms, we have that 
\[
[z^p]\left(\frac{1}{1-s_iz}F(z)\right)= 0
\]
for $i = 1, \ldots, r $ and wish to show that $r=1$ and $s_1=0$. 

We first claim that at least one $s_i$ is $0$. Suppose for the sake of contradiction that $\prod_i s_i \ne 0$.  We recall
\[
a(z)=\frac{1}{\prod_{i=1}^{r}(1-s_iz)}F(z)
\]
as in Lemma~\ref{lem:resbyparts}.  Recall that $b_i(z)=\prod_{j \ne i} (1-s_jz)$ for $i=1,\dots,r$.   Now, for any $i$, we have $a(z)b_i(z)=\frac{1}{1-s_iz}F(z)$ so in particular 
\[
[z^p]a(z)b_i(z)=[z^p]\frac{1}{1-s_iz}F(z)=0.
\]
Therefore for any $\lambda_i \in k$, we have
\begin{equation}\label{eq:lincomb}
\sum_{i=1}^{r} \lambda_i[z^p]a(z)b_i(z)=[z^p]\left( a(z)\sum_{i=1}^{r}\lambda_ib_i(z)\right)=0.
\end{equation}
By Lemma~\ref{lem:vand}, $b_i(z)$ are linearly independent polynomials of degree at most $r-1$. Therefore for $0 \le m \le r-1$ we can choose $\{\lambda_i^m\}$ so that 
\[
\sum_i\lambda_i^mb_i(z)=z^m.
\]
Choosing these $\lambda_i^m$ in (\ref{eq:lincomb}), we see that
\[
[z^{p-r+1}]a(z)=\dots=[z^p]a(z)=0.
\]
By repeated application of Lemma~\ref{lem:resbyparts} starting with $l=p-r$, we find that 
\[
[z^l]a(z)=0 \text{ for $l=p-r,p-r-1,\dots,0$}.
\]
This implies that $[z^0]a(z)=0$, a contradiction. We conclude that $s_i=0$ for some $i$.

Now, suppose without loss of generality that $s_{r}=0$. If $r = 1$, we are done.  Otherwise, we have $\sum_{i=1}^{r-1} m_is_i = \sum_{i=1}^{r} m_is_i = 0$ and because the $s_i$ are distinct we see that $\prod_{i=0}^{r-1} s_i \ne 0$.  Therefore, we may apply Lemma~\ref{lem:resbyparts} and repeat the argument above with $s_1, \ldots, s_{r - 1}$.  We find that at least one of the $s_i$ with $i < r$ is $0$, a contradiction since the $s_i$ are distinct. Therefore, we must have $r=1$ and $s_1=0$, concluding the proof.
\end{proof}

\section{Proof of the main result}

\begin{theorem}\label{thm:main}
For generic $c$, the $f_i$ are linearly independent and generate the maximal ideal $J$ of the polynomial representation for $\HH_{1, c}(\h)$, and the irreducible quotient $L = A/J$ is a complete intersection with Hilbert series 
\[
h_L(t) = \left(\frac{1-t^p}{1-t}\right)^{n-1}.
\]
\end{theorem}
\begin{proof}
By \cite[Proposition 3.4]{BC1}, the Hilbert series of $L$ is 
\[
h_L(t) = \left(\frac{1-t^p}{1-t}\right)^{n-1}h(t^p)
\]
for some polynomial $h(t)$ with nonnegative integer coefficients.  On the other hand, by Propositions~\ref{prop:linind} and \ref{prop:ci}, $A/I$ is a complete intersection with $n-1$ linearly independent generators $f_i$ in degree $p$.  Therefore, its Hilbert series is 
\[
h_{A/I}(t)=\left(\frac{1-t^p}{1-t}\right)^{n-1}.
\]
By Proposition~\ref{prop:ann}, the generators $f_i$ of $I$ are singular vectors, so $I \subseteq J$, implying that $h_{A/I}(t) \ge h_{A/J}(t)$ coefficient-wise.  We conclude that $h(t) = 1$, hence $h_{A/I}(t)=h_{A/J}(t)$ and $I=J$, completing the proof.
\end{proof}

\bibliographystyle{alpha}
\bibliography{rca-bib}

\end{document}
