\documentclass{article}
\usepackage{amsmath}
\usepackage{amsfonts}
\usepackage{amssymb,amsthm}
\usepackage{color}
\usepackage[margin=1in]{geometry}
\usepackage{hyperref}
\usepackage{graphicx}
\usepackage{verbatim}

\numberwithin{equation}{section}
\newtheorem{theorem}[equation]{Theorem}
\newtheorem{proposition}[equation]{Proposition}
\newtheorem{lemma}[equation]{Lemma}
\newtheorem{corollary}[equation]{Corollary}
\newtheorem{conjecture}[equation]{Conjecture}
\newtheorem{problem}[equation]{Problem}
\newtheorem{remark}[equation]{Remark}
\newtheorem{definition}[equation]{Definition}
\newenvironment{defn}[1][]{%
    \begin{definition}[#1]\pushQED{\qed}}{\popQED \end{definition}}
\parindent=0in
\parskip=0.15in

\newcommand{\xx}[1]{\textcolor{blue}{#1}}
\newcommand{\zb}{\overline{z}}
\newcommand{\cc}[1]{\overline{#1}}
\newcommand{\m}[1]{\left|#1\right|}
\newcommand{\Res}{\operatorname{Res}}
\newcommand{\p}[1]{\left(#1\right)}
\newcommand{\cis}{\text{cis}\,}
\newcommand{\sgn}{\text{sgn}}
\newcommand{\D}{\triangle}
\newcommand{\ZZ}{\mathbb{Z}}
\newcommand{\f}[1]{\sum_{n=-\infty}^\infty #1_ne^{inx}}
\newcommand{\fn}[1]{\sum_{n=-\infty}^\infty #1e^{inx}}
\newcommand{\BB}{\mathcal{B}}
\newcommand{\M}{\mathcal{M}}
\newcommand{\mm}[1]{\m{\m{#1}}}
\newcommand{\ang}[1]{\left\langle #1 \right\rangle}


\begin{document}

\section{$p=3, 3 \mid n$, $n-1$ case full argument}

\subsection{Generators}

Let $x_1,\dots,x_{n-1}$ be a basis for the $n-1$ dimensional representation of $S_n$ derived from the standard representation with basis $e_0,\dots,e_{n-1}$ by the relation $x_i=e_i-e_0$ and the action of the symmetric group defined accordingly. Let $s_{ij}$ be the transposition of $i$ and $j$, for $0 \le i < j \le n-1$. 

For $1 \le i < j \le n-1$, the relevant eigenvectors are $x_i-x_j$. For $s_{0i}$ for $1 \le i \le n-1$, the relevant eigenvector is $x_i$. 

Let $f_a$ for $a=1,\dots, n-1$ be $f_a = -x_a^3+c \left(\sum_{i=1}^{n-1} x_ax_i(x_a-x_i)\right)+c^2 \left(\sum_{i=1}^{n-1} \sum_{j=1}^{n-1} x_ix_jx_a\right)$. We will show that all the $f_a$ are killed by Dunkl operators. Because of their symmetry, we need only consider $f_1$. Since $f_1$ is symmetric in the $x_i$ excluding $x_1$, the only Dunkl operators we need consider are $D_1,D_2$.

We note that $f_1$ is preserved by $s_{ij}$ for all $i,j >1$. We consider $s_{1a}$ for $2 \le a$. We see that $s_{1a}f_1=-x_a^3+c \left(\sum_{i=1}^{n-1} x_ax_i(x_a-x_i)\right)+c^2 \left(\sum_{i=1}^{n-1} \sum_{j=1}^{n-1} x_ix_jx_a\right)=f_a$. Then:

\begin{align*}
f_1-f_a&=-(x_1-x_a)^3+c \left(\sum_{i=1}^{n-1} x_1x_i(x_1-x_i)-x_ax_i(x_a-x_i)\right)+c^2 \left(\sum_{i=1}^{n-1} \sum_{j=1}^{n-1} x_ix_j(x_1-x_a)\right)\\
&=-(x_1-x_a)^3+c \left(\sum_{i=1}^{n-1} x_i(x_1+x_a-x_i)(x_1-x_a)\right)+c^2 \left(\sum_{i=1}^{n-1} \sum_{j=1}^{n-1} x_ix_j(x_1-x_a)\right)\\
\end{align*}

Then $\frac{f_1-s_{1a}f_1}{x_1-x_a}=-(x_1-x_a)^2+c \left(\sum_{i=1}^{n-1} x_i(x_1+x_a-x_i)\right)+c^2 \left(\sum_{i=1}^{n-1} \sum_{j=1}^{n-1} x_ix_j\right)$.

We consider $s_{0a}$ for $2 \le a$. We see that $s_{0a}$ sends $x_a$ to $-x_a$ and $x_i$ for $i \ne a$ to $x_i-x_a$. We see that: 

\begin{align*}
s_{0a}f_1&=s_{0a}\left(-x_1^3+c \left(\sum_{i=1}^{n-1} x_1x_i(x_1-x_i)\right)+c^2 \left(\sum_{i=1}^{n-1} \sum_{j=1}^{n-1} x_ix_jx_1\right)\right)\\
&=s_{0a}\left(-x_1^3+c \left(\sum_{i \ne 1,a} x_1x_i(x_1-x_i)\right)+cx_1x_a(x_1-x_a)+c^2 \left(\sum_{i\ne a}\sum_{j \ne a}x_ix_jx_1\right)+2c^2\left(\sum_{i \ne a} x_ix_ax_1\right)+c^2x_a^2x_1\right)\\
&=-(x_1-x_a)^3+c \left(\sum_{i \ne 1,a} (x_1-x_a)(x_i-x_a)(x_1-x_i)\right)+c(x_1-x_a)(-x_a)(x_1)+\\&c^2 \left(\sum_{i\ne a}\sum_{j \ne a}(x_i-x_a)(x_j-x_a)(x_1-x_a)\right)+2c^2\left(\sum_{i \ne a} (x_i-x_a)(-x_a)(x_1-x_a)\right)+c^2(-x_a)^2(x_1-x_a)\\\end{align*}\begin{align*}
&=-x_1^3+x_a^3+c \left(\sum_{i \ne 1,a} x_1(x_i-x_a)(x_1-x_i)\right)-c \left(\sum_{i \ne 1,a} x_a(x_a-x_i)(x_i-x_1)\right)+c(x_1-x_a)(-x_a)(x_1)+\\&c^2 \left(\sum_{i\ne a}\sum_{j \ne a}(x_i-x_a)(x_j-x_a)(x_1-x_a)\right)+2c^2\left(\sum_{i \ne a} (x_i-x_a)(-x_a)(x_1-x_a)\right)+c^2(-x_a)^2(x_1-x_a)\\
&=-x_1^3+x_a^3+c \left(\sum_{i \ne 1,a} x_1x_i(x_1-x_i)\right)-c \left(\sum_{i \ne 1,a} x_ax_i(x_a-x_i)\right)+c\left(\sum_{i \ne 1,a}-x_1^2x_a+x_a^2x_1\right)+c(x_1-x_a)(-x_a)(x_1)+\\&c^2 \left(\sum_{i\ne a}\sum_{j \ne a}(x_i-x_a)(x_j-x_a)(x_1-x_a)\right)+2c^2\left(\sum_{i \ne a} (x_i-x_a)(-x_a)(x_1-x_a)\right)+c^2(-x_a)^2(x_1-x_a)\\
&=-x_1^3+x_a^3+c \left(\sum_{i \ne 1,a} x_1x_i(x_1-x_i)\right)-c \left(\sum_{i \ne 1,a} x_ax_i(x_a-x_i)\right)
-cx_1x_a(x_1-x_a)+\\&c^2 \left(\sum_{i\ne a}\sum_{j \ne a}(x_i-x_a)(x_j-x_a)(x_1-x_a)\right)+2c^2\left(\sum_{i \ne a} (x_i-x_a)(-x_a)(x_1-x_a)\right)+c^2(-x_a)^2(x_1-x_a)\\
&=-x_1^3+x_a^3+c \left(\sum_{i \ne 1,a} x_1x_i(x_1-x_i)\right)-c \left(\sum_{i \ne 1,a} x_ax_i(x_a-x_i)\right)
+cx_1x_a(x_1-x_a)-cx_1x_a(x_a-x_1)+\\&c^2 \left(\sum_{i\ne a}\sum_{j \ne a}(x_i-x_a)(x_j-x_a)(x_1-x_a)\right)+2c^2\left(\sum_{i \ne a} (x_i-x_a)(-x_a)(x_1-x_a)\right)+c^2(-x_a)^2(x_1-x_a)\\
&=-x_1^3+x_a^3+c \left(\sum_{i=1}^{n-1} x_1x_i(x_1-x_i)\right)-c \left(\sum_{i=1}^{n-1} x_ax_i(x_a-x_i)\right)
+\\&c^2 \left(\sum_{i\ne a}\sum_{j \ne a}(x_i-x_a)(x_j-x_a)(x_1-x_a)\right)+2c^2\left(\sum_{i \ne a} (x_i-x_a)(-x_a)(x_1-x_a)\right)+c^2(-x_a)^2(x_1-x_a)\\
&=-x_1^3+x_a^3+c \left(\sum_{i=1}^{n-1} x_1x_i(x_1-x_i)\right)-c \left(\sum_{i=1}^{n-1} x_ax_i(x_a-x_i)\right)
+\\&c^2 \left(\sum_{i\ne a}\sum_{j \ne a}(x_ix_j-x_ax_j-x_ax_i+x_a^2)(x_1-x_a)\right)+2c^2\left(\sum_{i \ne a} (x_a^2-x_ax_i)(x_1-x_a)\right)+c^2x_a^2(x_1-x_a)\\
&=-x_1^3+x_a^3+c \left(\sum_{i=1}^{n-1} x_1x_i(x_1-x_i)\right)-c \left(\sum_{i=1}^{n-1} x_ax_i(x_a-x_i)\right)
+\\&c^2 \left(\sum_{i\ne a}\sum_{j \ne a}(x_ix_j+x_a^2)(x_1-x_a)\right)+c^2\left(\sum_{i \ne a} x_ax_i(x_1-x_a)\right)+c^2\left(\sum_{i \ne a} (x_a^2-x_ax_i)(x_1-x_a)\right)+c^2x_a^2(x_1-x_a)\\
&=-x_1^3+x_a^3+c \left(\sum_{i=1}^{n-1} x_1x_i(x_1-x_i)\right)-c \left(\sum_{i=1}^{n-1} x_ax_i(x_a-x_i)\right)
+\\&c^2 \left(\sum_{i\ne a}\sum_{j \ne a}(x_ix_j+x_a^2)(x_1-x_a)\right)+c^2\left(\sum_{i \ne a} x_a^2(x_1-x_a)\right)+c^2x_a^2(x_1-x_a)\\\end{align*}\begin{align*}
&=-x_1^3+x_a^3+c \left(\sum_{i=1}^{n-1} x_1x_i(x_1-x_i)\right)-c \left(\sum_{i=1}^{n-1} x_ax_i(x_a-x_i)\right)
+c^2 \left(\sum_{i\ne a}\sum_{j \ne a}(x_ix_j)(x_1-x_a)\right)+\\&2c^2\left(\sum_{i \ne a} x_a^2(x_1-x_a)\right)+c^2x_a^2(x_1-x_a)\\
&=-x_1^3+x_a^3+c \left(\sum_{i=1}^{n-1} x_1x_i(x_1-x_i)\right)-c \left(\sum_{i=1}^{n-1} x_ax_i(x_a-x_i)\right)
+c^2 \left(\sum_{i\ne a}\sum_{j \ne a}(x_ix_j)(x_1-x_a)\right)+\\& 2c^2x_a^2(x_1-x_a)+c^2x_a^2(x_1-x_a)\\
&=-x_1^3+x_a^3+c \left(\sum_{i=1}^{n-1} x_1x_i(x_1-x_i)\right)-c \left(\sum_{i=1}^{n-1} x_ax_i(x_a-x_i)\right)
+c^2 \left(\sum_{i\ne a}\sum_{j \ne a}(x_ix_j)(x_1-x_a)\right)\\
&=-x_1^3+x_a^3+c \left(\sum_{i=1}^{n-1} x_1x_i(x_1-x_i)\right)-c \left(\sum_{i=1}^{n-1} x_ax_i(x_a-x_i)\right)
+c^2 \left(\sum_{i\ne a}\sum_{j \ne a}x_ix_jx_1-x_ix_jx_a)\right)\\
&=f_1-f_a
\end{align*} 

Then $\frac{f_1-s_{0a}f_1}{x_a}=f_a/x_a=-x_a^2-c \left(\sum_{i=1}^{n-1} x_i(x_i-x_a)\right)+c^2 \left(\sum_{i=1}^{n-1} \sum_{j=1}^{n-1} x_ix_j\right)$. 

The only remaining case to consider is $s_{01}$. We see that $s_{01}$ sends $x_1$ to $-x_1$ and $x_a$ for $a \ge 2$ to $x_a-x_1$. Then $s_{01}f_1$ goes to 

\begin{align*}
&-(-x_1)^3+c \left(\sum_{i=2}^{n-1} (-x_1)(x_i-x_1)(-x_1-(x_i-x_1))\right)+c^2 \left(\sum_{i=2}^{n-1} \sum_{j=2}^{n-1} (x_i-x_1)(x_j-x_1)(-x_1)\right)+\\&2c^2\left(\sum_{i=2}^{n-1}(x_i-x_1)x_1^2\right)+c^2(-x_1)^3\\
&=x_1^3+c \left(\sum_{i=2}^{n-1} x_1x_i(x_i-x_1)\right)+c^2 \left(\sum_{i=2}^{n-1} \sum_{j=2}^{n-1}-x_ix_jx_1+x_ix_1^2+x_jx_1^2-x_1^3\right)+\\&2c^2\left(\sum_{i=2}^{n-1}(x_i-x_1)x_1^2\right)+2c^2x_1^3\\
&=x_1^3+c \left(\sum_{i=2}^{n-1} x_1x_i(x_i-x_1)\right)+c^2 \left(\sum_{i=1}^{n-1} \sum_{j=1}^{n-1}-x_ix_jx_1\right)\\  
&=-f_1\\
\end{align*}

The above used the fact that $n-2 \equiv 1 \mod 3$. Therefore $\frac{f_1-s_{01}f_1}{x_1}=2f_1/x_1=-f_1/x_1=x_1^2+c \left(\sum_{i=1}^{n-1} x_i(x_i-x_1)\right)-c^2 \left(\sum_{i=1}^{n-1} \sum_{j=1}^{n-1} x_ix_j\right)$.

We can use these to calculate the effects of the Dunkl operators. 

\subsection{Dunkl operators}

\subsubsection{$D_1f_1$}

We first consider $D_1f_1=\partial_1f_1-c\sum_{a > 1} \frac{f_1-s_{1a}f_1}{x_1-x_a}-c\frac{f_1-s_{01}f_1}{x_1}$. Let $G_1=\sum_{a > 1} \frac{f_1-s_{1a}f_1}{x_1-x_a}, G_2=\frac{f_1-s_{01}f_1}{x_1}$. We wish to show that $\partial_1f_1=cG_1+cG_2$, since then $D_1f_1=0$. We calculate $\partial_1f_1$: 

\begin{align*}
\partial_1f_1&=\partial_1\left(-x_1^3+c \left(\sum_{i=1}^{n-1} x_1x_i(x_1-x_i)\right)+c^2 \left(\sum_{i=1}^{n-1} \sum_{j=1}^{n-1} x_ix_jx_1\right)\right)\\
&=c \left(\sum_{i=2}^{n-1} x_1x_i+x_i(x_1-x_i)\right)+c^2 \left(\sum_{i=2}^{n-1} \sum_{j=2}^{n-1} x_ix_j\right)+2c^2 \left(\sum_{i=2}^{n-1} 2x_ix_1\right)\\
&=c \left(\sum_{i=2}^{n-1} x_1x_i+x_ix_1-x_i^2\right)+c^2 \left(\sum_{i=2}^{n-1} \sum_{j=2}^{n-1} x_ix_j\right)+c^2 \left(\sum_{i=2}^{n-1} x_ix_1\right)\\
&=c \left(\sum_{i=2}^{n-1} -x_1x_i-x_i^2\right)+c^2 \left(\sum_{i=2}^{n-1} \sum_{j=1}^{n-1} x_ix_j\right)\\
\end{align*}

We also see that $G_1= \sum_{a > 1} \frac{f_1-s_{1a}f_1}{x_1-x_a} = \sum_{a=2}^{n-1} -(x_1-x_a)^2+c \left(\sum_{i=1}^{n-1} x_i(x_1+x_a-x_i)\right)+c^2 \left(\sum_{i=1}^{n-1} \sum_{j=1}^{n-1} x_ix_j\right)$ and $G_2=x_1^2+c \left(\sum_{i=1}^{n-1} x_i(x_i-x_1)\right)-c^2 \left(\sum_{i=1}^{n-1} \sum_{j=1}^{n-1} x_ix_j\right)$.

Since $n-2 \equiv 1 \mod 3$, we see that the $c^2$ terms cancel in $G_1+G_2$. We also note that $x_i(x_i-x_1)=x_i(x_1+x_1-x_i)-x_i^2$ for all $i$, so $G_1+G_2= x_1^2-c\left(\sum_{i=1}^{n-1} x_i^2\right)+\sum_{a=1}^{n-1} -(x_1-x_a)^2+c \left(\sum_{i=1}^{n-1} x_i(x_1+x_a-x_i)\right)$.

\begin{align*}
G_1+G_2&= x_1^2-c\left(\sum_{i=1}^{n-1} x_i^2\right)+\sum_{a=1}^{n-1} -(x_1-x_a)^2+c \left(\sum_{i=1}^{n-1} x_i(x_1+x_a-x_i)\right)\\
&= x_1^2-c\left(\sum_{i=1}^{n-1} x_i^2\right)+\sum_{a=1}^{n-1} -x_1^2-x_a^2+2x_ax_1+c \left(\sum_{i=1}^{n-1} x_i(x_1+x_a-x_i)\right)\\
&= x_1^2-c\left(\sum_{i=1}^{n-1} x_i^2\right)+\sum_{j=1}^{n-1} -x_1^2-x_j^2+2x_jx_1+c \left(\sum_{i=1}^{n-1} x_i(x_1+x_j-x_i)\right)\\
&= x_1^2+-(n-1)x_1^2-c\left(\sum_{i=1}^{n-1} x_i^2\right)+\sum_{j=1}^{n-1} -x_j^2-x_jx_1+c \left(\sum_{i=1}^{n-1} x_i(x_1+x_j-x_i)\right)\\
&= 2x_1^2-c\left(\sum_{i=1}^{n-1} x_i^2\right)+\sum_{j=1}^{n-1} \left(-x_j^2-x_jx_1\right)+c \left(\sum_{i=1}^{n-1} \sum_{j=1}^{n-1} x_ix_1+x_ix_j-x_i^2\right)\\
&= \sum_{j=2}^{n-1} \left(-x_j^2-x_jx_1\right)-c\left(\sum_{i=1}^{n-1} x_i^2\right)+c \left(\sum_{i=1}^{n-1} \sum_{j=1}^{n-1} x_ix_1+x_ix_j-x_i^2\right)\\
&= \sum_{j=2}^{n-1} \left(-x_j^2-x_jx_1\right)-c\left(\sum_{i=1}^{n-1} x_i^2\right)+c \left(\sum_{i=1}^{n-1} \sum_{j=1}^{n-1}x_ix_j\right)-c \left(\sum_{i=1}^{n-1} x_ix_1-x_i^2\right)\\
&= \sum_{j=2}^{n-1} \left(-x_j^2-x_jx_1\right)-c\left(\sum_{i=1}^{n-1} x_i^2\right)+c \left(\sum_{i=1}^{n-1} \sum_{j=2}^{n-1}x_ix_j\right)+c \left(\sum_{i=1}^{n-1} x_i^2\right)\\
&= \sum_{j=2}^{n-1} \left(-x_j^2-x_jx_1\right)+c \left(\sum_{i=1}^{n-1} \sum_{j=2}^{n-1}x_ix_j\right)\\
\end{align*}

Then a simple change of indices tells us that $cG_1+cG_2=\partial_1f_1$ as desired.

\subsubsection{$D_2f_1$}

We see that $D_2f_1 = \partial_2f_1-c\frac{f_1-s_{12}f_1}{x_2-x_1} -c\frac{f_1-s_{02}f_1}{x_2}$. 

We calculate $\partial_2f_1$:

\begin{align*}
\partial_2f_1&=\partial_2\left(-x_1^3+c \left(\sum_{i=1}^{n-1} x_1x_i(x_1-x_i)\right)+c^2 \left(\sum_{i=1}^{n-1} \sum_{j=1}^{n-1} x_ix_jx_1\right)\right)\\
&=\partial_2\left(-x_1^3+c \left(\sum_{i=3}^{n-1} x_1x_i(x_1-x_i)\right)+cx_1x_2(x_1-x_2)+c^2 \left(\sum_{i \ne 2}\sum_{j \ne 2} x_ix_jx_1\right)+2c^2\left(\sum_{i \ne 2}x_ix_1x_2\right)+c^2x_2^2x_1\right)\\
&=cx_1^2-2cx_1x_2+2c^2\left(\sum_{i \ne 2}x_ix_1\right)+2c^2x_2x_1\\
&=cx_1^2+cx_1x_2+2c^2\left(\sum_{i=1}^{n-1}x_ix_1\right)\\
\end{align*}

Let $G_1=\frac{f_1-s_{12}f_1}{x_2-x_1}, G_2=\frac{f_1-s_{02}f_1}{x_2}$. If $c(G_1+G_2)=\partial_2f_1$, then $D_2f_1=0$.

\begin{align*}
G_1+G_2&=\frac{f_1-s_{12}f_1}{x_2-x_1}+\frac{f_1-s_{02}f_1}{x_2}\\
&=(x_1-x_2)^2-c \left(\sum_{i=1}^{n-1} x_i(x_1+x_2-x_i)\right)-c^2 \left(\sum_{i=1}^{n-1} \sum_{j=1}^{n-1} x_ix_j\right)-
x_2^2-c \left(\sum_{i=1}^{n-1} x_i(x_i-x_2)\right)+c^2 \left(\sum_{i=1}^{n-1} \sum_{j=1}^{n-1} x_ix_j\right)\\
&=x_1^2+x_1x_2+x_2^2-c \left(\sum_{i=1}^{n-1} x_i(x_1+x_2-x_i)\right)-
x_2^2-c \left(\sum_{i=1}^{n-1} x_i(x_i-x_2)\right)\\
&=x_1^2+x_1x_2-c \left(\sum_{i=1}^{n-1} x_i(x_1+x_2-x_i)\right)-c \left(\sum_{i=1}^{n-1} x_i(x_i-x_2)\right)\\
&=x_1^2+x_1x_2-c \left(\sum_{i=1}^{n-1} x_i(x_1+x_2-x_i+x_i-x_2)\right)\\
&=x_1^2+x_1x_2-c \left(\sum_{i=1}^{n-1} x_ix_1\right)\\
\end{align*}

Then we see easily that $\partial_2f_1=cG_1+cG_2$, so $D_2f_1=0$ as desired. 



\subsection{Complete intersection}

Then we see that the ideal $I$ generated by $f_1,\dots,f_{n-1}$ has $I \subseteq J$ where $J$ is the kernel of the $\beta$ form, since the generators $f_1,\dots,f_{n-1}$ are killed by the Dunkl operators. 

We note that the $f_i$ are linearly independent, since $x_i^3$ has nonzero coefficient in $f_j$ only when $i=j$. We will show that $A/I$, where $A=k[x_1,\dots,x_{n-1}]$, is a complete intersection by showing that $f_1(x) = \dots = f_{n-1}(x)=0$ implies $x=0$. 

Recall $f_a = -x_a^3+c \left(\sum_{i=1}^{n-1} x_ax_i(x_a-x_i)\right)+c^2 \left(\sum_{i=1}^{n-1} \sum_{j=1}^{n-1} x_ix_jx_a\right)$.

We also note that $\left(\sum_{i=1}^{n-1} x_i \right)^2= \left(\sum_{i=1}^{n-1} \sum_{j=1}^{n-1} x_ix_j\right)$, and $\left(\sum_{i=1}^{n-1} x_i \right)^2+\sum_{1 \le i < j \le n-1} x_ix_j = \sum_{i=1}^{n-1} x_i^2$ modulo $3$. 

Therefore:

\begin{align*}
f_a &= -x_a^3+c \left(\sum_{i=1}^{n-1} x_ax_i(x_a-x_i)\right)+c^2 \left(\sum_{i=1}^{n-1} \sum_{j=1}^{n-1} x_ix_jx_a\right)\\
&= -x_a^3+c \left(\sum_{i=1}^{n-1} x_a^2x_i-x_ax_i^2\right)+c^2 \left(\sum_{i=1}^{n-1} \sum_{j=1}^{n-1} x_ix_jx_a\right)\\
&= -x_a^3+cx_a^2 \left(\sum_{i=1}^{n-1} x_i\right) -cx_a\left( \sum_{i=1}^{n-1} x_i^2\right)+c^2x_a \left(\sum_{i=1}^{n-1} \sum_{j=1}^{n-1} x_ix_j\right)\\
&= -x_a^3+cx_a^2 \left(\sum_{i=1}^{n-1} x_i\right) +2cx_a\left(\sum_{i=1}^{n-1} x_i \right)^2+2cx_a\left(\sum_{1 \le i < j \le n-1} x_ix_j\right)+c^2x_a\left(\sum_{i=1}^{n-1} x_i \right)^2\\
&=-x_a^3+c x_a^2\left(\sum_{i=1}^{n-1} x_i \right)+(c^2+2c)x_a\left(\sum_{i=1}^{n-1} x_i\right)^2+x_a\left(\sum_{1 \le i < j  \le n-1} 2cx_ix_j\right)
\end{align*}

We see that $f_a=-x_a^3+c x_a^2\left(\sum_{i=1}^{n-1} x_i \right)+(c^2+2c)x_a\left(\sum_{i=1}^{n-1} x_i\right)^2+x_a\left(\sum_{1 \le i < j  \le n-1} 2cx_ix_j\right)$. 


We now consider the sum of the $f_a$:

\begin{align*}
\sum_{a=1}^{n-1} f_a &= \sum_{a=1}^{n-1}\left(-x_a^3+c x_a^2\left(\sum_{i=1}^{n-1} x_i \right)+(c^2+2c)x_a\left(\sum_{i=1}^{n-1} x_i\right)^2+x_a\left(\sum_{1 \le i < j  \le n-1} 2cx_ix_j\right)\right)\\
 &= \left(\sum_{a=1}^{n-1}-x_a^3\right)+c\left(\sum_{a=1}^{n-1}x_a^2\right)\left(\sum_{i=1}^{n-1} x_i \right)+(c^2+2c)\left(\sum_{a=1}^{n-1}x_a\right)\left(\sum_{i=1}^{n-1} x_i\right)^2+\left(\sum_{a=1}^{n-1}x_a\right)\left(\sum_{1 \le i < j  \le n-1} 2cx_ix_j\right)\\
  &= 2\left(\sum_{i=1}^{n-1}x_i^3\right)+c\left(\sum_{i=1}^{n-1}x_i^2\right)\left(\sum_{i=1}^{n-1} x_i \right)+(c^2+2c)\left(\sum_{i=1}^{n-1}x_i\right)\left(\sum_{i=1}^{n-1} x_i\right)^2+\left(\sum_{i=1}^{n-1}x_i\right)\left(\sum_{1 \le i < j  \le n-1} 2cx_ix_j\right)\\
    &= 2\left(\sum_{i=1}^{n-1}x_i^3\right)+c\left(\left(\sum_{i=1}^{n-1} x_i \right)^2+\left(\sum_{1 \le i < j \le n-1} x_ix_j\right) \right)\left(\sum_{i=1}^{n-1} x_i \right)+(c^2+2c)\left(\sum_{i=1}^{n-1}x_i\right)^3+\\&2c\left(\sum_{i=1}^{n-1}x_i\right)\left(\sum_{1 \le i < j  \le n-1} x_ix_j\right)\\
        &= 2\left(\sum_{i=1}^{n-1}x_i^3\right)+c\left(\sum_{i=1}^{n-1} x_i \right)^3+c\left(\sum_{1 \le i < j \le n-1} x_ix_j\right)\left(\sum_{i=1}^{n-1} x_i \right)+(c^2+2c)\left(\sum_{i=1}^{n-1}x_i\right)^3+\\&2c\left(\sum_{i=1}^{n-1}x_i\right)\left(\sum_{1 \le i < j  \le n-1} x_ix_j\right)\\
        &=(c^2+2)\left(\sum_{i=1}^{n-1} x_i\right)^3\\
\end{align*}


We see that $\sum_{a=1}^{n-1} f_a=  (c^2+2)\left(\sum_{i=1}^{n-1} x_i \right)^3$.


Therefore if $f_1(x) = \dots = f_{n-1}(x)=0$, we see that $\sum_{i=1}^{n-1} x_i = 0$. Modulo this new relation, we see that $f_a=x_a\left(-x_a^2+\sum_{1 \le i < j  \le n-1} 2cx_ix_j\right)$. Therefore if $f_a=0$ for all $a$, we see that for each $a$, either $x_a=0$ or $x_a$ is a square root of $C=\sum_{1 \le i < j  \le n-1} 2cx_ix_j$. If $C=0$ or $C$ has no square roots, then all the $x_a$ are 0 and we are done; so we assume $C$ has two square roots which are additive inverses of each other. Then each pair $x_ix_j$ either multiply to $\pm C$ or $0$, so $C=2cmC$ for some integer $m$. Then $(1+cm)C=0$; if $C \ne 0$, then $1+cm=0$; however, $m$ is an integer, so this is impossible. Therefore $C=0$, so all the $x_a$ are $0$.

Therefore $f_1(x) = \dots = f_{n-1}(x)=0$ implies $x=0$, so $A/I$ is a complete intersection; it then must have Hilbet polynomial $h_{A/I}(t)=(t^2+t+1)^{n-1}$. 

By Proposition 3.4 in http://arxiv.org/abs/1107.0504, we see that the Hilbert polynomial of $A/J$ is $(t^2+t+1)^{n-1}h(t^3)$ for some polynomial $h$ with nonnegative integer coefficients; since $I \subseteq J$, we see that $h_{A/I}(t) \ge h_{A/J}(t)$ coefficientwise; however by this restriction of the form of $h_{A/J}(t)$, we see that the only possible choice for $h$ is $h(t)=1$. Therefore $h_{A/I}(t)=h_{A/J}(t)$, so $I=J$ and these $n-1$ generators generate the whole ideal.


\section{$p \mid n$, $n-1$ case full argument}

We assume $p > 2$ since $p=2$ has been fully characterized. 

\subsection{Generators}

Let $x_0,\dots,x_{n-2}$ be a basis for the $n-1$ dimensional representation of $S_n$ derived from the standard representation with basis $e_0,\dots,e_{n-1}$ by taking the quotient by $e_0+\dots+e_{n-1}$; assume $x_i$ is the representative for $e_i$ in the quotient. We see that we can in fact say that the representation is spanned by $x_0,\dots,x_{n-1}$ with the relation $x_0+\dots+x_{n-1}=0$, or $x_{n-1}=-x_0-\dots-x_{n-2}$. Let $s_{ij}$ be the transposition of $i$ and $j$ for $0 \le i < j \le n-1$. For $0 \le i < j \le n-1$ the relevant eigenvectors are $x_i-x_j$.

We let $g=\prod_{j=0}^{n-1} (1-x_jz)$, where $z$ is another variable.

Let $f_a$ for $a=0,\dots,n-1$ be the formal power series in $z$ defined by $f_a=\frac{1}{1-x_az}\left(\sum_{k=0}^{p-1} \binom{c}{k} (g-1)^k\right)$ where $\binom{c}{k}=\frac{c(c-1)\dots(c-k+1)}{k!}$. We will show that the coefficient of $z^p$ in $f_a$ is killed by the Dunkl operators for all $a$. Since the Dunkl operators consist of taking derivatives in the $x_i$, dividing by polynomials in the $x_i$, linear operations, and the action of the symmetric group on the $x_i$, we see that we can just apply the Dunkl operators to the $f_a$ and check that the coefficient of $z^p$ in the resulting formal power series is $0$. 

Let $y_0,\dots,y_{n-1}$ be a basis for the dual of the standard representation of $S_n$. Then the dual of the $(n-1)$-dimensional quotient we are considering is spanned by $y_i-y_0$ for $0 < i \le n-1$. Therefore if $D_i$ is the Dunkl operator corresponding to $y_i$ acting on the standard representation of $S_n$, we see that the Dunkl operators on our quotient representation are $D_i-D_0$ for $0 < i \le n-1$

Because of the symmetry in the $f_a$, we need only consider the action of the Dunkl operators on $f_0$. Then again by symmetry we need only consider the action of $D_1-D_0$ on $f_0$. 

We also note that we can add powers of $z$ greater than $z^p$ at any stage of taking the Dunkl operator, since those will not affect the final result. 





\subsection{Dunkl operators}

We note $z^2 \mid g-1$ since $x_0+\dots+x_{n-1}=0$ divides the coefficient of $z$ in $g$. Therefore when $p > 2$ we see that $z^{p+1} \mid z^{2p-2} \mid (g-1)^{p-1}$. Therefore we can add multiples of $(g-1)^{p-1}$ at any stage, since even when multipled by other power series it cannot contribute anything to the coefficient of $z^p$. 

Since $x_{n-1}=-x_0-\dots-x_{n-2}$, we see that $\frac{\partial g}{\partial x_i}=-\frac{zg}{1-x_iz}+\frac{zg}{1-x_{n-1}z}$ for all $0 \le i < n-1$. Let $F=f_0(1-x_1z)=\left(\sum_{k=0}^{p-1} \binom{c}{k} (g-1)^k\right)$. Note that $F$ is symmetric. Then we see that for all $0 \le i < n-1$:

\begin{align*}
\frac{\partial F}{\partial x_i}&=\frac{\partial}{\partial x_i}\left(\sum_{k=0}^{p-1} \binom{c}{k} (g-1)^k\right)\\
&=\left(\sum_{k=1}^{p-1}k\binom{c}{k}(g-1)^{k-1}\right)\frac{\partial g}{\partial x_i}\\
&=\left(-\frac{z}{1-x_iz}+\frac{z}{1-x_{n-1}z}\right)\left(\sum_{k=1}^{p-1}k\binom{c}{k}(g-1)^{k-1}\right)g\\
&=\left(-\frac{z}{1-x_iz}+\frac{z}{1-x_{n-1}z}\right)\left(\sum_{k=0}^{p-2}(k+1)\binom{c}{k+1}(g-1)^{k}\right)g\\
&=\left(-\frac{z}{1-x_iz}+\frac{z}{1-x_{n-1}z}\right)\left(\sum_{k=0}^{p-2}c\binom{c-1}{k}(g-1)^{k}\right)(g-1+1)\\
&=\left(-\frac{z}{1-x_iz}+\frac{z}{1-x_{n-1}z}\right)\left(\left(\sum_{k=0}^{p-2}c\binom{c-1}{k}(g-1)^{k}\right)+\left(\sum_{k=0}^{p-2}c\binom{c-1}{k}(g-1)^{k+1}\right)\right)\\
&=\left(-\frac{z}{1-x_iz}+\frac{z}{1-x_{n-1}z}\right)\left(\left(\sum_{k=0}^{p-2}c\binom{c-1}{k}(g-1)^{k}\right)+\left(\sum_{k=1}^{p-1}c\binom{c-1}{k-1}(g-1)^{k}\right)\right)\\
&\text{(We see that for }k=1,\dots,p-1\text{ we have }\binom{c-1}{k}+\binom{c-1}{k-1}=\binom{c}{k},\text{ that }\binom{c-1}{0}=\binom{c}{0}\\&\text{ as a polynomial, and that we can add a multiple of }(g-1)^{p-1}).\\
&=\left(-\frac{z}{1-x_iz}+\frac{z}{1-x_{n-1}z}\right)\left(\sum_{k=0}^{p-1}c\binom{c}{k}(g-1)^{k}\right)\\
&=\left(-\frac{zc}{1-x_iz}+\frac{zc}{1-x_{n-1}z}\right)F\\
\end{align*}

We also see that:

\begin{align*}
\frac{\partial f_0}{\partial x_1}&=\frac{\partial}{\partial x_1}\left(\frac{1}{1-x_0z}\left(\sum_{k=0}^{p-1} \binom{c}{k} (g-1)^k\right)\right)\\
&=\frac{1}{1-x_0z}\frac{\partial F}{\partial x_1}\\
&=\frac{1}{1-x_0z}\left(-\frac{zc}{1-x_1z}+\frac{zc}{1-x_{n-1}z}\right)F\\
&=\left(-\frac{zc}{1-x_1z}+\frac{zc}{1-x_{n-1}z}\right)f_0\\
\end{align*}

and that:

\begin{align*}
\frac{\partial f_0}{\partial x_0}&=\frac{\partial}{\partial x_1}\left(\frac{1}{1-x_0z}\left(\sum_{k=0}^{p-1} \binom{c}{k} (g-1)^k\right)\right)\\
&=\frac{z}{(1-x_0z)^2}F+\frac{1}{1-x_0z}\frac{\partial F}{\partial x_0}\\
&=\frac{z}{1-x_0z}f_0+\frac{1}{1-x_0z}\left(-\frac{zc}{1-x_0z}+\frac{zc}{1-x_{n-1}z}\right)F\\
&=\frac{z}{1-x_0z}f_0+\left(-\frac{zc}{1-x_0z}+\frac{zc}{1-x_{n-1}z}\right)f_0\\
&=\left(\frac{z(1-c)}{1-x_0z}+\frac{zc}{1-x_{n-1}z}\right)f_0\\
\end{align*}

We note that $f_0$ is invariant under $s_{ij}$ where $0 < i,j$. Therefore for transpositions we need only consider transpositions of the form $s_{0i}$ for $0 < i \le n-1$. 

\begin{align*}
\frac{1-s_{0i}}{x_0-x_i}(f_0)&=\frac{1}{x_0-x_i}\left(\frac{F}{1-x_0z}-\frac{F}{1-x_iz}\right)\\
&=\frac{1}{x_0-x_i}\left(\frac{1}{1-x_0z}-\frac{1}{1-x_iz}\right)F\\
&=\frac{1}{x_0-x_i}\left(\frac{(1-x_iz)-(1-x_0z)}{(1-x_0z)(1-x_iz)}\right)F\\
&=\frac{x_0z-x_iz}{(1-x_0z)(1-x_iz)(x_0-x_i)}F\\
&=\frac{z}{(1-x_iz)(1-x_0z)}F\\
&=\frac{z}{1-x_iz}f_0\\
\end{align*}


We recall that we need only consider the action of $D_1-D_0$ on $f_0$. We consider $D_0f_0, D_1f_0$ separately first. We see that $D_0=\left(\frac{\partial}{\partial x_0}-c\sum_{j > 0} \frac{1-s_{0j}}{x_0-x_j}\right), D_1=\left(\frac{\partial}{\partial x_1}-c \frac{1-s_{01}}{x_1-x_0}\right)$ since $f_0$ is invariant under $s_{ij}$ where $0 < i,j$. 

\begin{align*}
D_0f_0&=\left(\frac{\partial}{\partial x_0}-c\sum_{j > 0} \frac{1-s_{0j}}{x_0-x_j}\right)(f_0)\\
&=\frac{\partial f_0}{\partial x_0}-c\sum_{j > 0} \frac{1-s_{0j}}{x_0-x_j}(f_0)\\
&=\left(\frac{z(1-c)}{1-x_0z}+\frac{zc}{1-x_{n-1}z}\right)f_0-\sum_{j > 0} \frac{zc}{1-x_jz}f_0\\
\end{align*}
\begin{align*}
D_1f_0&=\left(\frac{\partial}{\partial x_1}-c \frac{1-s_{01}}{x_1-x_0}\right)(f_0)\\
&=\frac{\partial f_0}{\partial x_1}+c \frac{1-s_{01}}{x_0-x_1}(f_0)\\
&=\left(-\frac{zc}{1-x_1z}+\frac{zc}{1-x_{n-1}z}\right)f_0+\frac{zc}{1-x_1z}f_0\\
&=\frac{zc}{1-x_{n-1}z}f_0\\
\end{align*}

It is then easy to see that $(D_1-D_0)(f_0)=\frac{z(c-1)}{1-x_0z}f_0+\sum_{j>0} \frac{zc}{1-x_jz}f_0$. 

In order to show that the $p$th coefficient in this formal power series is $0$, we must consider $\frac{\partial f_0}{\partial z}$. 

We see easily that $\frac{\partial g}{\partial z} = g \sum_j \frac{-x_j}{1-x_jz}$. We now consider $\frac{\partial F}{\partial z}$:

\begin{align*}
\frac{\partial F}{\partial z}&=\frac{\partial}{\partial z}\left(\sum_{k=0}^{p-1} \binom{c}{k} (g-1)^k\right)\\
&=\left(\sum_{k=1}^{p-1}k\binom{c}{k}(g-1)^{k-1}\right)\frac{\partial g}{\partial z}\\
&=\left(\sum_j \frac{-x_j}{1-x_jz}\right)\left(\sum_{k=1}^{p-1}k\binom{c}{k}(g-1)^{k-1}\right)g\\
&\text{Note that above we showed that }\left(\sum_{k=1}^{p-1}k\binom{c}{k}(g-1)^{k-1}\right)g=\left(\sum_{k=0}^{p-1}c\binom{c}{k}(g-1)^{k}\right)\\&\text{ up to the addition of multiples of }z^{p+1}.\\
&=\left(\sum_j \frac{-x_j}{1-x_jz}\right)\left(\sum_{k=0}^{p-1}c\binom{c}{k}(g-1)^{k}\right)\\
&=\left(\sum_j \frac{-cx_j}{1-x_jz}\right)F
\end{align*}

From this it follows that:

\begin{align*}
\frac{\partial f_0}{\partial z} &= \frac{\partial}{\partial z}\left(\frac{F}{1-x_0z}\right)\\
&=\frac{1}{1-x_0z} \frac{\partial F}{\partial z}+\frac{x_0}{1-x_0z} F\\
&=\frac{1}{1-x_0z}\left(\sum_j \frac{-cx_j}{1-x_jz}\right)F+\frac{x_0}{(1-x_0z)^2} F\\
&=\left(\sum_j \frac{-cx_j}{1-x_jz}\right)f_0+\frac{x_0}{1-x_0z} f_0\\
\end{align*}

We now again consider $(D_1-D_0)(f_0)$. Recall that $n \equiv 0 \bmod p$, so in particular we can add $n$ times any multiple of $f_1$ since that is $0$ in characteristic $p$. 

\begin{align*}
(D_1-D_0)(f_0)&=\frac{z(c-1)}{1-x_0z}f_0+\sum_{j>0} \frac{zc}{1-x_jz}f_0\\
&=-\frac{z}{1-x_0z}f_0+\sum_{j} \frac{zc}{1-x_jz}f_0\\
&=-\frac{z}{1-x_0z}f_0++\left(\sum_{j} \frac{zc}{1-x_jz}f_0\right)-nzcf_0\\
&=-zf_0+zf_0-\frac{z}{1-x_0z}f_0+\left(\sum_{j} -zcf_0+\frac{zc}{1-x_jz}f_0\right)\\
&=-zf_0+\frac{z-x_0z^2}{1-x_0z}f_0-\frac{z}{1-x_0z}f_0+\left(\sum_{j} \frac{-zc+x_jcz^2}{1-x_jz}f_0+\frac{zc}{1-x_jz}f_0\right)\\
&=-zf_0+\frac{-x_0z^2}{1-x_0z}f_0+\left(\sum_{j} \frac{x_jcz^2}{1-x_jz}f_0\right)\\
&=-zf_0-z^2\left(\frac{x_0}{1-x_0z}f_0+\left(\sum_{j}- \frac{x_jc}{1-x_jz}f_0\right)\right)\\
&=-zf_0-z^2\frac{\partial f_0}{\partial z}
\end{align*}

Let $b$ be the coefficient of $z^{p-1}$ in $f_0$. We see that the coefficient of $z^p$ in $-zf_0$ is $-b$. Then the coefficient of $z^{p-2}$ in $\frac{\partial f_0}{\partial z}$ is $(p-1)b=-b$, so the coefficient of $z^p$ in $-z^2\frac{\partial f_0}{\partial z}$ is $b$. Therefore the coefficient of $z^p$ in $-zf_0-z^2\frac{\partial f_0}{\partial z}$ is $-b+b=0$, so the coefficient of $z^p$ in $(D_1-D_0)(f_0)$ is $0$. Then if $B$ is the coefficient of $z^p$ in $f_0$, it is clear that $(D_1-D_0)(B)=0$ as desired. By the symmetry of the $f_a$, this is the only Dunkl operator we need consider; it is clear that the coefficients of $z^p$ in all the $f_a$ are killed by the Dunkl operators.

\subsection{Complete intersection}

Let $B_a$ be the coefficient of $z^p$ in $f_a$. We see that $f_a=\frac{1}{1-x_az}\left(\sum_{k=0}^{p-1} \binom{c}{k} (g-1)^k\right)=\left(\sum_{k=0}^\infty x_a^kz^k\right)\left(\sum_{k=0}^{p-1} \binom{c}{k} (g-1)^k\right)$. It is then clear that $B_a$ is a homogeneous polynomial in the $x_i$ of degree $p$, since the coefficient of $z^k$ for any $k$ is a homogeneous polynomial in the $x_i$ of degree $k$ for all $k$ (this follows from the fact that this is true in both multiplicands). 

We also note that:

\begin{align*}
\sum_{a=0}^{n-1} f_a&=\left(\sum_a \frac{1}{1-x_az}\right)F\\
&=\left(\sum_a \frac{1}{1-x_az}\right)F-nF\\
&=\left(\sum_a\frac{x_az-1}{1-x_az}+ \frac{1}{1-x_az}\right)F\\
&=\left(\frac{-x_az}{1-x_az}\right)F\\
&=z\frac{\partial F}{\partial z}
\end{align*}

Then the coefficient of $z^p$ in this sum is the coefficient of $z^{p-1}$ in $\frac{\partial F}{\partial z}$, which is $p$ times the coefficient of $z^p$ in $F$, which must be $0$ since we are in characteristic $p$. The coefficient of $z^p$ in this sum is also $\sum_a B_a$, so we have $\sum_a B_a=0$. 




\end{document}